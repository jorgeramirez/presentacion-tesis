\section{Definici\'on del Problema}

\begin{frame}{Definici\'on del Problema}
Las interfaces mediante voz presentan ciertas caracter{\'\i}sticas que las
distinguen de las interfaces visuales \cite{GabrielVoice2007}:

\begin{itemize}
    \item Transitoriedad: la voz desaparece tan pronto como se termina de pronunciar una oraci\'on,
    lo cual obliga a recordar lo que se dijo. Las interfaces visuales, por otro lado, son persistentes.
    \item Invisibilidad: la voz no es visible, lo cual hace dif{\'\i}cil indicar al usuario las opciones
    disponibles y los comandos necesarios para ejecutarlas. En las interfaces visuales los men\'ues
    cumplen esta funci\'on.
    \item Asimetr{\'\i}a: la voz puede producirse r\'apidamente, pero comprender lo que se escucha requiere
    m\'as tiempo. As{\'\i}, un usuario puede hablar m\'as r\'apido de lo que escribe con un teclado; sin embargo,
    al usuario le toma m\'as tiempo comprender lo que escucha que lo que lee.
\end{itemize}
\end{frame}

\begin{frame}{Definici\'on del Problema (2)}
Estas caracter{\'\i}sticas suponen un desaf{\'\i}o adicional al momento de dise\~nar interfaces mediante voz del
usuario. A\'un as{\'\i}, estas interfaces poseen un gran potencial en situaciones en las cuales la
combinaci\'on tradicional de teclado, rat\'on y monitor resulta problem\'atica \cite{NielsenVoice2003}:

\begin{itemize}
    \item Usuarios con discapacidades, las cuales les impiden manejar apropiadamente el rat\'on y/o
    el teclado o visualizar la informaci\'on en el monitor.
    \item Usuarios en situaciones de manos y vista ocupadas: como la conducci\'on de un veh{\'\i}culo o
    la reparaci\'on de equipamiento complejo.
    \item Usuarios sin acceso a un teclado o monitor: en este caso los usuarios podr{\'\i}an acceder
    a un sistema a trav\'es de un tel\'efono convencional.
\end{itemize}

\end{frame}

\begin{frame}{Definici\'on del Problema (3)}
El problema del dise\~no de una interfaz por voz del usuario plantea varias cuestiones importantes para
su estudio. En particular, la interacción entre usuario y computadora mediante la voz podría mejorarse
a través de la investigación de la influencia de factores como:  


\begin{itemize}
    \item El dominio de la aplicaci\'on con la cual se interactúa.
    \item El tama\~no del lenguaje o la cantidad de comandos utilizados.
    \item La longitud de los comandos o la cantidad de palabras que componen un comando.
    \item La duraci\'on de la interacci\'on entre usuario y computadora.
\end{itemize}


\end{frame}

\begin{frame}{Definici\'on del Problema (4)}
De modo a contrastar la pr\'actica con el conocimiento te\'orico adquirido,
se diseña e implementa una aplicación simple de composición musical que ofrece una
interfaz mediante voz del usuario.
El programa de composici\'on musical controlado por voz es un medio para:

\begin{itemize}
    \item Explorar y experimentar con el reconocimiento del habla y las interfaces de usuario.
    \item Contrastar teor{\'\i}a y pr\'actica.
    \item Demostrar la factibilidad de la soluci\'on de un problema real utilizando el
    reconocimiento del habla.
\end{itemize}
\end{frame}
