\section{Definici\'on del Problema}

\begin{frame}{Definici\'on del Problema}

Las interfaces mediante voz presentan ciertas caracter{\'\i}sticas que las
distinguen de las interfaces visuales \cite{GabrielVoice2007}:

\begin{itemize}
    \item Transitoriedad
    \item Invisibilidad
    \item Asimetr{\'\i}a
\end{itemize}
A\'un as{\'\i}, estas interfaces poseen un gran potencial en situaciones en las cuales la
combinaci\'on tradicional de teclado, rat\'on y monitor resulta problem\'atica \cite{NielsenVoice2003}:

\begin{itemize}
    \item Usuarios con discapacidades
    \item Usuarios en situaciones de manos y vista ocupadas
    \item Usuarios sin acceso a un teclado o monitor
\end{itemize}

\end{frame}

\begin{frame}{Definici\'on del Problema (2)}
La interacción entre usuario y computadora mediante la voz podría mejorarse
a través de la investigación de la influencia de factores como:

\begin{itemize}
    \item El dominio de la aplicaci\'on con la cual se interactúa.
    \item El tama\~no del lenguaje o la cantidad de comandos utilizados.
    \item La longitud de los comandos o la cantidad de palabras que componen un comando.
    \item La duraci\'on de la interacci\'on entre usuario y computadora.
\end{itemize}

De modo a contrastar la pr\'actica con el conocimiento te\'orico adquirido,
se diseña e implementa una aplicación simple de composición musical que ofrece una
interfaz mediante voz del usuario.

\end{frame}