\section{Conclusiones}

\begin{frame}{Conclusiones}

\begin{itemize}
    \item Diseño de la Interfaz
        \begin{itemize}
            \item La naturalidad del lenguaje es de gran importancia para la interfaz.
            \item Interactuar con la aplicaci\'on, no con la interfaz gr\'afica.
            \item Utilizar el sonido como medio de retroalimentaci\'on.
        \end{itemize}
    \item Implementación de la Interfaz
        \begin{itemize}
            \item Seleccionar las herramientas de acuerdo al \mbox{proyecto}.
            \item Considerar la posibilidad de errores en el \emph{software}.
            \item Realizar pruebas y modificaciones tempranas.
        \end{itemize}
\end{itemize}

\end{frame}

\begin{frame}{Conclusiones}
\framesubtitle{Prueba con Usuarios}

\begin{itemize}
    \item Correlación
        \begin{itemize}
            \item Errar es humano: Los usuarios prefieren probar (y fallar) a leer.
            \item Considerar la memoria del usuario al evaluar los resultados.
            \item Validar hip\'otesis triviales.
        \end{itemize}
    \item Error Humano
        \begin{itemize}
            \item Evitar comandos de m\'as de 4 palabras.
            \item El contexto de los comandos no es lo (\'unico) que importa.
            \item Dar la debida importancia al entrenamiento del usuario.
            \item Considerar factores humanos como la fatiga.
            \item Reconsiderar los comandos con tasa de error elevada.
        \end{itemize}
    \item Encuesta
        \begin{itemize}
            \item Opini\'on positiva de los usuarios hacia \emph{TamTam Listens} y el reconocimiento del habla.
            \item Los puntos a mejorar est\'an en las respuestas de los usuarios.
            \item Preferir aplicaciones poco interactivas para interfaces por voz.
        \end{itemize}
    \end{itemize}

\end{frame}