\section{Conclusiones}

\begin{frame}{Conclusiones}

\begin{table}[ht]
  \begin{tabular}{|p{2.75cm}|p{8cm}|}
    \hline
    Tema & Conclusi\'on \\
    \hline
    \multirow{3}{2.75cm}{\textbf{Dise\~no de la Interfaz}} & La naturalidad del lenguaje es de gran importancia \mbox{para la interfaz.} \\
    \hhline{~-}
    &Interactuar con la aplicaci\'on, no con la interfaz \mbox{gr\'afica.}\\
    \hhline{~-}
    &Utilizar el sonido como medio de \mbox{retroalimentaci\'on.} \\
    \thickhline
    \multirow{3}{2.75cm}{\textbf{Implementaci\'on de la Interfaz}} & Seleccionar las herramientas de acuerdo al \mbox{proyecto}.\\
    \hhline{~-}
    & Considerar la posibilidad de errores en el \emph{software}.\\
    \hhline{~-}
    & Realizar pruebas y modificaciones tempranas.\\
    \hline
  \end{tabular}
\end{table}


% \begin{itemize}
%     \item Dise\~no de la Interfaz
%         \begin{itemize}
%             \item La naturalidad del lenguaje es de gran importancia para la interfaz.
%             \item Interactuar con la aplicaci\'on, no con la interfaz gr\'afica.
%             \item Utilizar el sonido como medio de retroalimentaci\'on.
%         \end{itemize}
%     \item Implementaci\'on de la Interfaz
%         \begin{itemize}
%             \item Seleccionar las herramientas de acuerdo al \mbox{proyecto}.
%             \item Considerar la posibilidad de errores en el \emph{software}.
%             \item Realizar pruebas y modificaciones tempranas.
%         \end{itemize}
% \end{itemize}

\end{frame}

\begin{frame}{Conclusiones}
\framesubtitle{Prueba con Usuarios}

\begin{table}[ht]
  \begin{tabular}{|p{2cm}|p{8.5cm}|}
    \hline
    Tema & Conclusi\'on \\
    \hline
    \multirow{3}{2cm}{\textbf{Correlaci\'on}} & Errar es humano: Los usuarios prefieren probar \mbox{(y fallar) a leer.} \\
    \hhline{~-}
    &Considerar la memoria del usuario al evaluar los \mbox{resultados.}\\
    \hhline{~-}
    &Validar hip\'otesis triviales. \\
    \thickhline
    \multirow{5}{2cm}{\textbf{Error Humano}} & Evitar comandos de m\'as de 4 palabras.\\
    \hhline{~-}
    & Dar la debida importancia al entrenamiento del \mbox{usuario.}\\
    \hhline{~-}
    & Considerar factores humanos como la fatiga.\\
    \hhline{~-}
    & Reconsiderar los comandos con tasa de error elevada.\\
    \hline
  \end{tabular}
\end{table}

% \begin{itemize}
%     \item Correlaci\'on
%         \begin{itemize}
%             \item Errar es humano: Los usuarios prefieren probar (y fallar) a leer.
%             \item Considerar la memoria del usuario al evaluar los resultados.
%             \item Validar hip\'otesis triviales.
%         \end{itemize}
%     \item Error Humano
%         \begin{itemize}
%             \item Evitar comandos de m\'as de 4 palabras.
%             \item El contexto de los comandos no es lo (\'unico) que importa.
%             \item Dar la debida importancia al entrenamiento del usuario.
%             \item Considerar factores humanos como la fatiga.
%             \item Reconsiderar los comandos con tasa de error elevada.
%         \end{itemize}
%     \item Encuesta
%         \begin{itemize}
%             \item Opini\'on positiva de los usuarios hacia \emph{TamTam Listens} y el reconocimiento del habla.
%             \item Los puntos a mejorar est\'an en las respuestas de los usuarios.
%             \item Preferir aplicaciones poco interactivas para interfaces por voz.
%         \end{itemize}
%     \end{itemize}

\end{frame}

\begin{frame}{Conclusiones}
\framesubtitle{Prueba con Usuarios}

\begin{table}[ht]
  \begin{tabular}{|p{2cm}|p{8.5cm}|}
    \hline
    Tema & Conclusi\'on \\
    \hline
    \multirow{3}{2cm}{\textbf{Encuesta}} & Opini\'on positiva de los usuarios hacia \mbox{\emph{TamTam Listens} y el reconocimiento del habla.}\\
    \hhline{~-}
    & Los puntos a mejorar est\'an en las respuestas de los usuarios.\\
    \hhline{~-}
    & Preferir aplicaciones poco interactivas para interfaces por voz.\\
    \hline
  \end{tabular}
\end{table}
\end{frame}