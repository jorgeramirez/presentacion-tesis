\section{Conclusiones}

\begin{frame}{Conclusiones}
\framesubtitle{Dise\~no de la Interfaz}
\begin{itemize}
\item{\textbf{Interactuar con la aplicaci\'on, no con la interfaz gr\'afica\\}}
En base a la experiencia, resulta recomendable asociar un comando a una funcionalidad
y no a una \'unica acci\'on en la interfaz tradicional.

El reemplazar directamente clics y pulsaciones de teclas por comandos de voz da
como resultado una interfaz poco natural y dif{\'\i}cil de utilizar. Por tanto, para
\foreign{TamTam Listens}, fue necesario replantear por completo el modo de acceso a las 
funcionalidades de manera independiente a la interfaz gr\'afica.

A modo de ejemplo, para cambiar el volumen en \emph{TamTam Edit} se requiere presionar un 
bot\'on para desplegar un submen\'u y utilizar un \emph{slider}, mientras que en \emph{TamTam Listens}
es necesario un \'unico comando ``Aumentar/Disminuir Volumen''.

\end{itemize}
\end{frame}

\begin{frame}{Conclusiones (2)}
\framesubtitle{Dise\~no de la Interfaz}
\begin{itemize}

\item{\textbf{Interactuar con la aplicaci\'on, no con la interfaz gr\'afica\\}}
En base a la experiencia, resulta recomendable asociar un comando a una funcionalidad
y no a una \'unica acci\'on en la interfaz tradicional.

El reemplazar directamente clics y pulsaciones de teclas por comandos de voz da
como resultado una interfaz poco natural y dif{\'\i}cil de utilizar. Por tanto, para
\foreign{TamTam Listens}, fue necesario replantear por completo el modo de acceso a las 
funcionalidades de manera independiente a la interfaz gr\'afica.

A modo de ejemplo, para cambiar el volumen en \emph{TamTam Edit} se requiere presionar un 
bot\'on para desplegar un submen\'u y utilizar un \emph{slider}, mientras que en \emph{TamTam Listens}
es necesario un \'unico comando ``Aumentar/Disminuir Volumen''.
\end{itemize}
\end{frame}

\begin{frame}{Conclusiones (3)}
\framesubtitle{Dise\~no de la Interfaz}
\begin{itemize}

\item{\textbf{Utilizar el sonido como medio de retroalimentaci\'on\\}}
El protocolo de interacci\'on entre el usuario y la aplicaci\'on puede pensarse de forma similar a una 
conversaci\'on entre personas. 

El usuario accede a las funcionalidades a trav\'es de comandos de voz.
A su vez, la aplicaci\'on puede utilizar la voz, o el sonido en general, para comunicar un mensaje al usuario.

En el caso particular de \emph{TamTam Listens}, se utilizaron notificaciones de audio para confirmar
ciertas operaciones. Por ejemplo, al crear o editar una nota, se reproduce la misma a modo de
confirmaci\'on.
\end{itemize}
\end{frame}

\begin{frame}{Conclusiones (4)}
\framesubtitle{Dise\~no de la Interfaz}
\begin{itemize}
\item{\textbf{Utilizar el sonido como medio de retroalimentaci\'on\\}}
El protocolo de interacci\'on entre el usuario y la aplicaci\'on puede pensarse de forma similar a una 
conversaci\'on entre personas. 

El usuario accede a las funcionalidades a trav\'es de comandos de voz.
A su vez, la aplicaci\'on puede utilizar la voz, o el sonido en general, para comunicar un mensaje al usuario.

En el caso particular de \emph{TamTam Listens}, se utilizaron notificaciones de audio para confirmar
ciertas operaciones. Por ejemplo, al crear o editar una nota, se reproduce la misma a modo de
confirmaci\'on.
\end{itemize}
\end{frame}


\begin{frame}{Conclusiones (5)}
\framesubtitle{Implementaci\'on de la Interfaz}
\begin{itemize}

\item{\textbf{Seleccionar las herramientas de acuerdo al \mbox{proyecto}\\}}
La herramienta elegida tiene gran influencia sobre el posterior proceso de
desarrollo, por lo cual esta decisi\'on debe tomarse analizando las caracter{\'\i}sticas
propias del proyecto en cuesti\'on.

Algunas cuestiones que pueden tomarse en consideraci\'on son:

\begin{itemize}
    \item El tiempo del que se dispone.
    \item El dinero del que se dispone.
    \item El conocimiento t\'ecnico del equipo de desarrolladores.
    \item La plataforma sobre la cual debe ejecutarse el sistema.
    \item La necesidad de que el sistema funcione sin conexi\'on a internet.
    \item El soporte existente para el idioma que se busca reconocer.
\end{itemize}

La evaluaci\'on de varias opciones disponibles para la implementaci\'on de un sistema
basado en reconocimiento del habla, cuyos resultados se incluyen como parte de este
trabajo, permiti\'o realizar la selecci\'on de manera debidamente informada y justificada.

\end{itemize}
\end{frame}


\begin{frame}{Conclusiones (6)}
\framesubtitle{Implementaci\'on de la Interfaz}
\begin{itemize}

\item{\textbf{Considerar la posibilidad de errores en el \emph{software}\\}}
Las primeras pruebas de \foreign{TamTam Listens} arrojaron tasas de error y tiempos de
respuesta muy elevados, relacionados a un componente que se planeaba utilizar
para integrar \foreign{TamTam Edit} con \foreign{PocketSphinx}.

Luego de numerosos intentos fallidos, se opt\'o por desechar el componente e implementar una soluci\'on 
utilizando \emph{D-Bus} para la integraci\'on. 

La identificaci\'on de la causa del problema y la
b\'usqueda de una soluci\'on conllevaron un retraso significativo en la etapa de desarrollo.

\end{itemize}
\end{frame}

\begin{frame}{Conclusiones (7)}
\framesubtitle{Implementaci\'on de la Interfaz}
\begin{itemize}

\item{\textbf{Realizar pruebas y modificaciones tempranas\\}}
Una vez que se cuenta con un prototipo m{\'\i}nimo de la aplicaci\'on, la realizaci\'on de pruebas
preliminares puede ofrecer resultados interesantes para el mejoramiento de la 
interfaz por voz del usuario.

En las caso de \foreign{TamTam Listens}, las pruebas preliminares hicieron notorios dos
inconvenientes: la elevada tasa de error para los comandos de una o dos palabras
y el efecto negativo de las palabras fuera del vocabulario para la precisi\'on del sistema. 

Estos resultados obligaron a realizar modificaciones sobre el lenguaje inicialmente planteado, 
permitiendo disminuir la tasa de error del sistema previamente a las pruebas con usuarios finales.
\end{itemize}
\end{frame}

HOLAAAAAA

\begin{frame}{Conclusiones (8)}
\framesubtitle{Prueba con Usuarios: Correlaci\'on}
\begin{itemize}
\item{\textbf{Errar es humano: Los usuarios prefieren probar (y fallar) a leer\\}}
Como puede observarse en el cuadro \ref{sec:tabla-correlacion}, la tasa de acierto del 
sistema ($A$) result\'o relacionada positivamente a la tasa de error humano ($E_2$) y la cantidad total de
errores del usuario ($E_3$). 

As{{\'\i}} tambi\'en, según los resultados expuestos en el cuadro \ref{sec:tabla-correlacion-2}, 
la cantidad total de comandos que utiliz\'o el usuario ($U$) result\'o relacionada positivamente con
ambas medidas del error humano ($E_2$ y $E_3$) y con el acierto del sistema ($A$).

Las relaciones anteriormente mencionadas parecen sugerir que el error humano no es necesariamente
un indicador negativo, afirmaci\'on que parece reforzarse con la correlaci\'on positiva, aunque moderada
a d\'ebil, entre las medidas del error humano  ($E_2$ y $E_3$) y la correctitud de la \'ultima 
tarea ($C$). Este valor indica que, en general, los usuarios que se equivocaron m\'as completaron 
mejor la tarea cuatro.
\end{itemize}
\end{frame}

\begin{frame}{Conclusiones (9)}
\framesubtitle{Prueba con Usuarios: Correlaci\'on}
\begin{itemize}
\item{\textbf{Considerar la memoria del usuario al evaluar los resultados\\}}
Resulta notoria en el cuadro \ref{sec:tabla-correlacion} la relación negativa de la memoria del usuario 
($M$) con las medidas de error humano  ($E_2$ y $E_3$)  y la duraci\'on de las 
tareas ($T_{1+2}$ y $T_{3+4}$) , lo cual puede considerarse como un 
indicador de la importancia de este factor. 

Esto es, en general, los usuarios con mayor puntaje en el 
test de memoria tardaron menos en realizar las tareas y cometieron menos errores.
\end{itemize}
\end{frame}


\begin{frame}{Conclusiones (10)}
\framesubtitle{Prueba con Usuarios: Correlaci\'on}
\begin{itemize}
\item{\textbf{Validar hip\'otesis triviales\\}}
Los resultados para algunas de las correlaciones de ambos cuadros de la sección \ref{sec:correlacion} son,
de cierta manera, previsibles. 
Si los resultados obtenidos no coinciden con las expectativas,
puede representar una se\~nal de alerta de alg\'un problema con la interfaz o con el an\'alisis realizado.

Por ejemplo, los usuarios para quienes la tasa de acierto del sistema ($A$) fue mayor obtuvieron 
mejores resultados en las tareas ($C$). Adem\'as, aquellos usuarios con mayor tasa de error ($E_2$)
tardaron m\'as en terminar las tareas ($T_{1+2}$). Estas afirmaciones se verifican mediante las 
correlaciones expuestas en el cuadro \ref{sec:tabla-correlacion}
\end{itemize}
\end{frame}





\begin{frame}{Conclusiones (11)}
\framesubtitle{Prueba con Usuarios: An\'alisis del Error Humano}
\begin{itemize}
\item{\textbf{Evitar comandos de m\'as de 4 palabras\\}}
Los resultados del cuadro \ref{sec:error-longitud} muestran que la tasa de error var{{\'\i}}a 
poco para los comandos de 2, 3, 4 palabras y aumenta considerablemente para los comandos de 5 y 
6 palabras.

Por este motivo, se sugiere que los comandos de una interfaz por voz del usuario 
no superen la longitud de 4 palabras.

Sin embargo, cabe destacar que esto representa un desaf{{\'\i}}o adicional a tener en cuenta.
Durante la etapa de implementaci\'on, pudo constatarse que los comandos de poca longitud presentaban
a menudo tasas de errores superiores a la media del sistema.

Mantener el equilibrio entre error humano y error de la aplicaci\'on, de acuerdo a la longitud
de los comandos, podr{{\'\i}}a suponer un problema no trivial. 
\end{itemize}
\end{frame}

\begin{frame}{Conclusiones (12)}
\framesubtitle{Prueba con Usuarios: An\'alisis del Error Humano}
\begin{itemize}
\item{\textbf{El contexto de los comandos no es lo (\'unico) que importa\\}}
Los resultados de el cuadro \ref{sec:error-contexto}, contrariamente a lo que podr{\'\i}a esperarse,
sugieren que no existe correlaci\'on entre el contexto de los comandos y la tasa de error humano asociada.

El an\'alisis de los datos de la prueba de usabilidad indica que el nivel contextual de los comandos no
fue determinante para la tasa de error humano, aunque cabe destacar que esto no permite descartar
por completo la influencia de este factor sobre el rendimiento de los usuarios.


\end{itemize}
\end{frame}

\begin{frame}{Conclusiones (13)}
\framesubtitle{Prueba con Usuarios: An\'alisis del Error Humano}
\begin{itemize}
\item{\textbf{Dar la debida importancia al entrenamiento del usuario\\}}
En la figura \ref{figure:gerror-tiempo} puede observarse una mayor concentraci\'on del error humano
desde el inicio hasta el 40\% del tiempo transcurrido. Esto podr{{\'\i}}a atribuirse a un periodo de
aprendizaje de uso de la interfaz.

La recomendaci\'on en este caso es otorgar la debida importancia al entrenamiento previo del usuario,
el cual se considera como el principal medio para disminuir la tasa de error humano inicial.

La duraci\'on y el dise\~no de las actividades de entrenamiento son puntos a tener en cuenta
al presentar una interfaz mediante voz del usuario.

\end{itemize}
\end{frame}

\begin{frame}{Conclusiones (14)}
\framesubtitle{Prueba con Usuarios: An\'alisis del Error Humano}
\begin{itemize}
\item{\textbf{Considerar factores humanos como la fatiga\\}}
En la misma figura puede apreciarse un leve aumento en la tasa de error hacia el final de cada
sesi\'on, lo que podr{{\'\i}}a un efecto del agotamiento sobre el usuario utilizando la
interfaz mediante voz.


Esta informaci\'on representa una llamada de atenci\'on hacia los factores humanos de la interacci\'on.
Los efectos influencia del cansancio, la frustraci\'on ante el error e incluso el estado de \'animo
del usuario no deben menospreciarse, e incluso podr{\'\i}an considerarse como tema para estudios
posteriores. 

Otro punto importante a considerar es que las tareas aumentaban en complejidad a medida que
el sujeto avanzaba en el experimento.

\end{itemize}
\end{frame}

\begin{frame}{Conclusiones (15)}
\framesubtitle{Prueba con Usuarios: An\'alisis del Error Humano}
\begin{itemize}
\item{\textbf{Reconsiderar los comandos con tasa de error elevada\\}}
Por \'ultimo, las tasas de error humano y del sistema permiten identificar comandos 
problem\'aticos que pueden modificarse en una versi\'on siguiente de la aplicaci\'on.

A modo de ejemplo, los comandos con mayor tasa de error humano se exponen en el 
cuadro ~\ref{sec:tabla-lista-comandos-error}.
Estos comandos son candidatos a ser modificados de modo a mejorar la usabilidad de la
aplicaci\'on.
\end{itemize}
\end{frame}


\begin{frame}{Conclusiones (16)}
\framesubtitle{Prueba con Usuarios: Encuesta}
\begin{itemize}

\item{\textbf{Opini\'on positiva de los usuarios hacia \emph{TamTam Listens} y el reconocimiento del habla\\}}
Los resultados de la encuesta expuestos en la secci\'on \ref{sec:resultados-encuesta},
permiten conluir un elevado grado de satisfacci\'on de los usuarios con \foreign{TamTam Listens} y una
predisposici\'on positiva hacia las interfaces mediante voz en general.

\item{\textbf{Los puntos a mejorar est\'an en las respuestas de los usuarios\\}}
Algunas de las recomendaciones m\'as comunes realizadas por los usuarios fueron:

\begin{itemize}
    \item Reemplazar la palabra ``Duplicar'' por la palabra ``Copiar''. 
    \item Reemplazar la palabra ``Exportar'' por la palabra ``Guardar''.
    \item Eliminar la palabra ``Nueva'' de los comandos de creaci\'on.
\end{itemize}
\end{itemize}
\end{frame}

\begin{frame}{Conclusiones (17)}
\framesubtitle{Prueba con Usuarios: Encuesta}
\begin{itemize}

\item{\textbf{Preferir aplicaciones poco interactivas para una interfaz por voz\\}}
La mayor{{\'\i}}a de los usuarios expres\'o que utilizar{{\'\i}}a una interfaz mediante voz
preferentemente en casos con un bajo grado de interactividad y un lenguaje sencillo.
Por ejemplo, los usuarios expresaron que les gustar{{\'\i}}a manejar el televisor, la radio, 
el horno microondas, entre otros, mediante la voz.
\end{itemize}
\end{frame}
