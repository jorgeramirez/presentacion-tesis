\section{Proceso B\'asico del Reconocimiento del Habla}

\begin{frame}{Proceso B\'asico del Reconocimiento del Habla (1/8)}
\uncover<1-2>{
\begin{quote}
\emph{La b\'usqueda de la oraci\'on m\'as probable W perteneciente al lenguaje L, dada la entrada ac\'ustica 0.}
\end{quote}
}

\uncover<2-2>{
\begin{align}
\hat{W} = \argmax_{W \in L} \overbrace{P(O|W)}^\text{M. ac\'ustico}\overbrace{P(W)}^\text{M. de lenguaje}
\label{eq:asrFundamental}
\end{align}
}
\end{frame}

\begin{frame}{Proceso B\'asico del Reconocimiento del Habla (2/8)}

\begin{figure}[H] 
\centering
\includegraphics[width=0.8\textwidth]{./graphics/proceso_00.png}
\caption{Proceso del reconocimiento del habla. Traducido a partir de \protect\cite{VerenichASR}.}
\label{figure:proceso}
\end{figure}
\end{frame}

\begin{frame}{Proceso B\'asico del Reconocimiento del Habla (2/8)}

\begin{figure}[H] 
\centering
\includegraphics[width=0.8\textwidth]{./graphics/proceso_0.png}
\caption{Proceso del reconocimiento del habla. Traducido a partir de \protect\cite{VerenichASR}.}
\label{figure:proceso}
\end{figure}
\end{frame}

\begin{frame}{Proceso B\'asico del Reconocimiento del Habla (2/8)}

\begin{figure}[H]  
\centering
\includegraphics[width=0.8\textwidth]{./graphics/proceso_1.png}
\caption{Proceso del reconocimiento del habla. Traducido a partir de \protect\cite{VerenichASR}.}
\label{figure:proceso}
\end{figure}
\end{frame}

\begin{frame}{Proceso B\'asico del Reconocimiento del Habla (2/8)}

\begin{figure}[H] 
\centering
\includegraphics[width=0.8\textwidth]{./graphics/proceso_2.png}
\caption{Proceso del reconocimiento del habla. Traducido a partir de \protect\cite{VerenichASR}.}
\label{figure:proceso}
\end{figure}
\end{frame}

\begin{frame}{Proceso B\'asico del Reconocimiento del Habla (2/8)}

\begin{figure}[H] 
\centering
\includegraphics[width=0.8\textwidth]{./graphics/proceso_3.png}
\caption{Proceso del reconocimiento del habla. Traducido a partir de \protect\cite{VerenichASR}.}
\label{figure:proceso}
\end{figure}
\end{frame}

\begin{frame}{Proceso B\'asico del Reconocimiento del Habla (2/8)}

\begin{figure}[H] 
\centering
\includegraphics[width=0.8\textwidth]{./graphics/proceso.png}
\caption{Proceso del reconocimiento del habla. Traducido a partir de \protect\cite{VerenichASR}.}
\label{figure:proceso}
\end{figure}
\end{frame}
\begin{frame}{Proceso B\'asico del Reconocimiento del Habla (3/8)}
\framesubtitle{Fase 1: Extracci\'on de caracter{\'\i}sticas}
\uncover<1-2>{
\begin{figure}[H]
\centering
\includegraphics[width=0.4\linewidth]{./graphics/formants.png}
\caption{Representaci\'on del espectro en el cual pueden identificarse los picos espectrales o formantes 
\cite{Jurafsky}.}
\label{figure:formants}
\end{figure}
}

\uncover<2-2>{
\begin{figure}[H]
\centering
\only<2-2>{
\includegraphics[width=0.7\linewidth]{./graphics/spectrogram.png}
\caption{Representaci\'on de un espectrograma, puede verse como una colecci\'on de espectros  ubicados uno despu\'es de otro \cite{Jurafsky}.}}
\label{figure:spectrogram}
\end{figure}
}
\end{frame}

\begin{frame}{Proceso B\'asico del Reconocimiento del Habla (4/8)}
\framesubtitle{Fase 1: Extracci\'on de caracter{\'\i}sticas}

\begin{figure}[H] 
\centering
\includegraphics[width=0.8\textwidth]{./graphics/extraccion_0.png}
\caption{Fase de extracci\'on de caracter{\'\i}sticas. Gr\'afico basado en \cite{VerenichASR}.}
\label{figure:hmm}
\end{figure}
\end{frame}

\begin{frame}{Proceso B\'asico del Reconocimiento del Habla (4/8)}
\framesubtitle{Fase 1: Extracci\'on de caracter{\'\i}sticas}

\begin{figure}[H] 
\centering
\includegraphics[width=0.8\textwidth]{./graphics/extraccion.png}
\caption{Fase de extracci\'on de caracter{\'\i}sticas. Gr\'afico basado en \cite{VerenichASR}.}
\label{figure:hmm}
\end{figure}
\end{frame}


\begin{frame}{Proceso B\'asico del Reconocimiento del Habla (5/8)}
\framesubtitle{Fase 2: Decodificaci\'on: Modelo de Lenguaje}
\uncover<1-3>{Probabilidad de ocurrencia de una secuencia de palabras $x_1,x_2,\ldots,x_n$ para un lenguaje dado.}
\vspace*{2\baselineskip}
\begin{itemize}
    \uncover<2-3>{\item Basado en N-Gramas
        \small \begin{equation*}
            \hspace*{-1.5cm} 
            p(\text{el, hombre, corre}) = p(el \mid \text{\textless} s\text{\textgreater}) \, 
            p(\text{\emph{hombre}} \mid el) \, p(corre \mid \text{\emph{hombre}}) \, 
            p(\text{\textless} /s\text{\textgreater} \mid corre)
        \end{equation*}}
        \vspace*{1\baselineskip}
        \uncover<3-3>{\item Basado en Gram\'atica
        \begin{bnf*}
            \bnfprod{pregunta}
            {\bnfts{Cu\'al} \bnfsp \bnfts{es} \bnfsp \bnfts{la}  \bnfpn{info} \bnfsp  \bnfts{en} \bnfsp \bnfpn{ciudad}} \\
            \bnfprod{info}
            {\bnfts{temperatura} \bnfor \bnfts{presi\'on atmosf\'erica} \bnfor \bnfts{hora}} \\
            \bnfprod{ciudad}
            {\bnfts{Par{\'\i}s} \bnfor \bnfts{Nueva York} \bnfor \bnfts{Roma}}
        \end{bnf*}}
    \end{itemize}
\end{frame}

\begin{frame}{Proceso B\'asico del Reconocimiento del Habla (6/8)}
\framesubtitle{Fase 2: Decodificaci\'on - Modelo Ac\'ustico}

\setbeamercovered{transparent}
\uncover<1-9>{Probabilidad de una entrada ac\'ustica $O$ dada una secuencia de \mbox{palabras $W$.}}
\begin{columns}
\column{0.45\linewidth}
\onslide<2-9>{
\begin{itemize}
    \uncover<1-9>{\item Modelos Ocultos de Markov}
        \begin{itemize}
                \uncover<4-9>{\item Estados: $S$}
                \uncover<5-9>{\item Observaciones: $V$}
                \uncover<6-9>{\item Probabilidad inicial: $\pi$}
                \uncover<7-9>{\item Probabilidad de transici\'on: $a$}
                \uncover<8-9>{\item Probabilidad de observaci\'on: $b$}
        \end{itemize}
    \uncover<9-9>{\item Diccionario Fon\'etico}
\end{itemize}
}

\column{0.55\linewidth}
\begin{figure}[H]
\centering
\uncover<3-9>{
\only<3-9>{
\includegraphics[width=1\textwidth]{./graphics/hmm.png}
\caption{Representaci\'on de un modelo oculto de Markov.}}}
\label{figure:esquema-herramientas}
\end{figure}
\end{columns}


\end{frame}

\begin{frame}{Proceso B\'asico del Reconocimiento del Habla (7/8)}
\framesubtitle{Fase 2: Decodificaci\'on - B\'usqueda}

\setbeamercovered{transparent}
\begin{columns}
\column{0.30\linewidth}
\onslide<1-3>{
\begin{itemize}
    \uncover<2-3>{\item Algoritmo de Viterbi.}
    \uncover<3-3>{\item Algoritmo A*.}
\end{itemize}
}

\column{0.7\linewidth}
\begin{figure}[H]
\centering
\uncover<1-3>{
\includegraphics[width=0.8\textwidth]{./graphics/espacio.png}
\caption{Espacio de b\'usqueda para un lenguaje simple de cuatro palabras. Traducido a partir de \cite{RenalsSearch}.}}
\label{figure:espacio-busqueda}
\end{figure}
\end{columns}


\end{frame}

\begin{frame}{Proceso B\'asico del Reconocimiento del Habla (8/8)}
\framesubtitle{Fase 2: Decodificaci\'on}
\begin{figure}[H] 
\centering
\includegraphics[width=0.8\textwidth]{./graphics/decodificacion0.png}
\caption{Fase de decodificaci\'on. Gr\'afico basado en \cite{VerenichASR}.}
\label{figure:decoding}
\end{figure}
\end{frame}


\begin{frame}{Proceso B\'asico del Reconocimiento del Habla (8/8)}
\framesubtitle{Fase 2: Decodificaci\'on}
\begin{figure}[H] 
\centering
\includegraphics[width=0.8\textwidth]{./graphics/decodificacion.png}
\caption{Fase de decodificaci\'on. Gr\'afico basado en \cite{VerenichASR}.}
\label{figure:decoding}
\end{figure}
\end{frame}
