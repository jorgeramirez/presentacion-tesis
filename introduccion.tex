\section{Introducci\'on}

\begin{frame}{Introducci\'on}

El reconocimiento del habla, tambi\'en conocido como reconocimiento autom\'atico del habla,
es el proceso de convertir una se\~nal de voz en una secuencia de
palabras, mediante un algoritmo implementado program\'aticamente \cite{JaisalAReview2012}. 
Su integraci\'on con interfaces de usuario busca una interacci\'on humano-computadora m\'as
natural, de manera a superar las limitaciones existentes en el modelo convencional
de interacci\'on.

\end{frame}

\begin{frame}{Objetivo General}
Realizar un estudio del trasfondo hist\'{o}rico, los fundamentos te\'{o}ricos 
y el estado del arte del reconocimiento del habla de modo a comprender, describir 
e introducir esta \'{a}rea de investigaci\'{o}n. 
\end{frame}

\begin{frame}{Objetivos Espec\'ificos (1)}

\begin{itemize}
    \item Presentar y describir los antecedentes hist\'oricos y el estado del arte del reconocimiento del habla.
    
    \item Analizar, ordenar y caracterizar el proceso t\'{i}pico de un sistema de reconocimiento del habla, 
        incluyendo los aspectos te\'{o}ricos involucrados en cada paso del mismo.

    \item Evaluar y seleccionar las herramientas disponibles que permiten la implementaci\'{o}n de soluciones 
        relacionadas al reconocimiento del habla.
\end{itemize}

\end{frame}

\begin{frame}{Objetivos Espec\'ificos (2)}

\begin{itemize}
    
    \item Dise\~{n}ar e implementar una interfaz mediante voz del usuario de manera a aplicar y 
    contrastar en la pr\'{a}ctica los conocimientos te\'{o}ricos adquiridos.
    
    \item Evaluar la soluci\'{o}n implementada de modo a obtener datos cuantitativos y cualitativos que 
        permitan extraer conclusiones sobre la aplicabilidad del reconocimiento del habla a las interfaces 
        de usuario.
\end{itemize}

\end{frame}
