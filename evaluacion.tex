\section{Evaluaci\'on}

\begin{frame}{Evaluaci\'on}
\framesubtitle{Metodolog\'ia}

Las pruebas se realizan con 12 estudiantes universitarios sin experiencia previa 
con la aplicaci\'on, en un Laboratorio de Inform\'atica de la Facultad Poĺit\'ecnica de 
la Universidad Nacional de Asunci\'on.

Las características de cada sesión son:
\begin{itemize}
    \item Duración: aproximadamente una hora
    \item Participantes:
        \begin{itemize}
            \item Usuario
            \item Facilitador
            \item Observador
        \end{itemize}
    \item Fases:
        \begin{itemize}
            \item Test de Memoria
            \item Entrenamiento
            \item Tareas
            \item Encuesta de Usabilidad
            \item Recopilación y Análisis de Datos
        \end{itemize}   
\end{itemize}
\end{frame}


\begin{frame}{Evaluaci\'on (2)}
\framesubtitle{Análisis de Datos}
    \begin{itemize}
        \item Memoria del Usuario: Test de Memoria ($M$)
        \item Correctitud de la Aplicación
            \begin{itemize}
                \item Tasa de Acierto ($A$)
                \item Tasa de Error de Comandos ($E_1$)
            \end{itemize}
        \item Error Humano
            \begin{itemize}
                \item Tasa de Error Humano ($E_2$)
                    \begin{itemize}
                        \item Tasa de Error por Longitud del Comando
                        \item Tasa de Error por Nivel Contextual del Comando
                        \item Tasa de Error por Comando
                        \item Distribuci\'on del Error Humano por Etapas de la Sesi\'on 
                    \end{itemize}
                \item Cantidad de Errores ($E_3$)
            \end{itemize}
        \item Eficiencia
            \begin{itemize}
                \item Duración de Tareas Uno y Dos ($T_{1+2}$)
                \item Duración de Tareas Tres y Cuatro($T_{3+4}$)
                \item Cantidad de Comandos Utilizados ($U$)
                \item Correctitud de la Tarea Cuatro ($C$)
            \end{itemize}
        \item Satisfacción del Usuario: Encuesta de Usabilidad
    \end{itemize}
\end{frame}