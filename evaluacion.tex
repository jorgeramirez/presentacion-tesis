\section{Evaluaci\'on}

\begin{frame}{Evaluaci\'on}
\framesubtitle{Metodolog\'ia}
Las pruebas se realizan con 12 estudiantes universitarios sin experiencia previa 
con la aplicaci\'on, cantidad escogida seg\'un se recomienda en \cite{Hwang:2010}.

Las sesiones se llevan a cabo durante 2 semanas en un Laboratorio de Inform\'atica de 
la Facultad Poĺit\'ecnica de la Universidad Nacional de Asunci\'on, Paraguay.
En cada sesi\'on participan un usuario y dos responsables de la prueba.
\end{frame}

Cada sesión tiene una duración aproximada de una hora y consta de las siguientes
fases:

\begin{frame}{Evaluaci\'on (2)}
\framesubtitle{Fases de la Evaluaci\'on}
\begin{enumerate}
    \item Test de Memoria: basado en el Test de Aprendizaje Auditivo Verbal de Rey \cite{Lopez1998}.
    Consiste en registrar cuantas palabras puede recordar el usuario, de una lista predeterminada de 15,
    a lo largo de 5 repeticiones. 
    \item Entrenamiento: se presentan al usuario una serie de videos instructivos y el manual de la 
    aplicaci\'on.
    \item Tareas: el usuario realiza una secuencia de cuatro tareas con \foreign{\mbox{TamTam Listens}}.
    Las tareas se realizan con un nivel decreciente de asistencia al usuario.
    
    El c\'alculo de los valores de las m\'etricas relacionadas a la correctitud de la aplicaci\'on y al 
    error humano, se realiza en base a las tareas 3 y 4, por realizarse estas sin asistencia del 
    facilitador.
\end{enumerate}
\end{frame}


\begin{frame}{Evaluaci\'on (3)}
\framesubtitle{Fases de la Evaluaci\'on}
\begin{enumerate}

    \item Encuesta de Usabilidad: el usuario completa un breve cuestionario acerca de 
    \foreign{TamTam Listens} y las interfaces mediante voz en general.
    \item Recopilaci\'on y An\'alisis de Datos: se recopilan los datos de la sesi\'on para su posterior
    sumarizaci\'on y an\'alisis. Estos datos incluyen los resultados del test de memoria, las grabaciones de las tareas y las respuestas de la encuesta de usabilidad.
\end{enumerate}
\end{frame}


\begin{frame}{Evaluaci\'on (4)}
\framesubtitle{Factores analizados}
En base a los datos de las sesiones, se analizan los siguientes factores:

\begin{enumerate}
    \item Memoria del usuario:
    Se mide a trav\'es de los resultados de la \'ultima repetici\'on del test de memoria 
    ($\boldsymbol{M}$), siendo 12--13 el resultado promedio esperado.
    \item Correctitud de la aplicaci\'on:
    Se mide mediante la tasa de aciertos de comandos ($\boldsymbol{A}$),
    la raz\'on entre la cantidad de comandos correctamente reconocidos  y la cantidad de comandos correctamente pronunciados por el usuario.
    
    El inverso aditivo de $\boldsymbol{A}$, la tasa de error de comandos ($\boldsymbol{E_1})$, tambi\'en se
    incluye en los resultados.

    \item Error Humano:
    Se mide mediante la cantidad total de errores cometidos por el usuario ($\boldsymbol{E_3}$) y la tasa
    de error ($\boldsymbol{E_2}$), la cual se define como la raz\'on entre la cantidad de comandos incorrectos
    y la cantidad de comandos pronunciados.
\end{enumerate}
\end{frame}

\begin{frame}{Evaluaci\'on (5)}
\framesubtitle{Factores analizados}
En base a los datos de las sesiones, se analizan los siguientes factores:

\begin{enumerate}
    \item Eficiencia:
    Se mide utilizando el tiempo de duraci\'on en minutos de las tareas 1 y 2 ($\boldsymbol{T_{1+2}}$),
    el tiempo de duraci\'on en minutos de las tareas 3 y 4 ($\boldsymbol{T_{3+4}}$) y la cantidad total 
    de comandos diferentes utilizados por el usuario ($\boldsymbol{U}$).

    Adem\'as, se considera la correctitud de la tarea 4 ($\boldsymbol{C}$) como medida de la eficiencia. 
    Esta se define como la raz\'on entre la cantidad de operaciones correctamente realizadas y el total 
    de operaciones (72).

    \item Satisfacci\'on del Usuario: 
    Se registra la opini\'on del usuario en una escala de Likert \cite{Allen:2007} del 1 al 7,
    adem\'as de las sugerencias del mismo, con respecto a:
    \begin{itemize}
        \item Que tan adecuadas son palabras utilizadas.
        \item Que tan adecuados son los comandos utilizados.
        \item Que tan adecuada resulta la duraci\'on del entrenamiento.
        \item Que tan frecuentemente utilizar{\'\i}a las interfaces por voz del usuario.
    \end{itemize}
\end{enumerate}
\end{frame}


\begin{frame}{Evaluaci\'on (6)}
\framesubtitle{An\'alisis del Error Humano}
La tasa de error humano se analiza de manera m\'as detallada, realiz\'andose los siguientes c\'alculos:
\begin{enumerate}
    \item Tasa de Error Humano por Longitud del Comando: se calcula la tasa de error para las
    diferentes longitudes de comandos (2 a 6 palabras).
    \item Tasa de Error Humano por Nivel Contextual del Comando: se clasifican los comandos
    seg\'un el contexto de su aplicaci\'on en generales, de pista y de comp\'as. La tasa de error
    se calcula para cada categor{\'\i}a.
    \item Tasa de Error Humano por Comando: se calcula la tasa de error para cada comando pronunciado
    por al menos un usuario.
\end{enumerate} 
\end{frame}

\begin{frame}{Evaluaci\'on (7)}
\framesubtitle{An\'alisis del Error Humano}
La tasa de error humano se analiza de manera m\'as detallada, realiz\'andose los siguientes c\'alculos:
\begin{enumerate}
    \item Distribuci\'on del Error Humano con respecto al Tiempo transcurrido: para cada usuario, 
    se divide $\boldsymbol{T_{3+4}}$ en 10 etapas (de 10\% de la duraci\'on cada una).
    Para cada etapa se calcula el porcentaje de errores cometidos, con respecto al total de
    errores cometidos por el usuario ($\boldsymbol{E_3}$). 
    La distribuci\'on se obtiene calculando el promedio de los valores correspondientes a 
    cada uno de los usuarios.
\end{enumerate} 
\end{frame}
