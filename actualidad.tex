\section{Actualidad y Aplicaciones}

\begin{frame}{Actualidad y Aplicaciones}
El panorama actual del reconocimiento del habla se caracteriza por su aplicaci\'on
en diversos \'ambitos, algunos de los cuales se mencionan a continuaci\'on:

\begin{itemize}
	\item{\textbf{Medicina y Derecho:}}
    existen aplicaciones comerciales y trabajos de investigaci\'on enfocados a la transcripci\'on de registros 
    m\'edicos \cite{LaiMedSpeak1997, USATodayHospitals} y reportes legales \cite{FalavignaAutomatic2009, NuanceLegal}
	mediante reconocimiento del habla.

	\item{\textbf{Milicia:}}
	se destacan los trabajos de Beek \cite{BeekAn1977} y Weinstein \cite{WeinsteinOpportunities1991},
	en los cuales se clasifican las potenciales aplicaciones de tecnolog\'ias del habla en 
	cinco categor{\'\i}as: seguridad, mando y control, transmisi\'on de datos y comunicaci\'on, 
	procesamiento de voz distorsionada y aplicaciones para entrenamiento.
\end{itemize}
\end{frame}
\begin{frame}{Actualidad y Aplicaciones (2)}
\begin{itemize}
	\item{\textbf{Telefon\'ia:}}
	las aplicaciones de reconocimiento del habla buscan reducir los costos, como
	la automatizaci\'on de consulta de directorio, 
	o producir ganancias mediante nuevos modelos de servicio, como servicios bancarios
	y de servicios de reserva \cite{RabinerApplications1997}.

	\item{\textbf{Accesibilidad:}}
	las interfaces mediante voz poseen un gran potencial para usuarios con alg\'un tipo de discapacidad
	que les impida manejar apropiadamente el rat\'on y/o el teclado o visualizar la informaci\'on en el monitor.
	Varios ejemplos de este tipo de aplicaciones se mencionan en \cite{AnanthiSurvey2013}.
\end{itemize}
\end{frame}
\begin{frame}{Actualidad y Aplicaciones (3)}
\begin{itemize}
	\item{\textbf{Industria Automotriz:}}
	seg\'un \cite{Kirriemuir2003Speech}, el desarrollo de sistemas de reconocimiento del habla para autom\'oviles
	se divide en las siguientes \'areas: dispositivos manos libres, control
	de los dispositivos de navegaci\'on, interacci\'on con el sistema de control y sistemas de manejo por voz.

	\item{\textbf{Tel\'efonos Inteligentes e Interfaces Web:}}
	se destaca el surgimiento de los asistentes virtuales,
	como \emph{Google Now} \cite{GoogleNow} y \emph{Siri} \cite{AppleSiri}. Adem\'as,
	puede mencionarse la \emph{Web Speech API} \cite{GoogleWebSpeechAPI}, una propuesta de \emph{Google}
	a la \emph{W3C} que permite integrar reconocimiento y s\'intesis del habla a las aplicaciones web.
\end{itemize}
\end{frame}
\begin{frame}{Actualidad y Aplicaciones (4)}
\begin{itemize}
	\item{\textbf{Videojuegos y Dom\'otica:}}
	existen juegos comerciales que interactuan con el usuario
	mediante la voz, como \emph{Say-N-Play}\cite{SayNPlay}. En el caso de la dom\'otica, 
	est\'an disponibles productos que permiten controlar los artefactos del hogar mediante la voz, 
	como \emph{HAL} \cite{HAL}.
\end{itemize}
\end{frame}
