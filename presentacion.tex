\documentclass[compress]{beamer}
%\documentclass[ignorenonframetext,handout]{beamer}
%\setbeamercovered{transparent}
%\usepackage[ISO 8859-1]{inputenc}
\usepackage{default}

% para usar figuras devemos acrescentar
\usepackage{graphicx}
\usepackage[export]{adjustbox}
%\usepackage{graphics}
%\DeclareGraphicsExtensions{.pdf,.png,.jpg}
%\DeclareGraphicsExtensions{.jpg, .eps}
%\DeclareGraphicsRule{.jpg}{eps}{.jpg}{`jpeg2ps -h -r 600 #1}
\usepackage{tikz}
%\usetikzlibrary{arrows,backgrounds,coordinatesystems,3d,shapes,plotmarks,automata,calendar,er,
%folding,matrix,mindmap,patterns,petri,plothandlers,topaths,trees} 
\usetikzlibrary{positioning}
%\usepgflibrary{decorations.pathreplacing}
\usetikzlibrary{decorations.pathreplacing}
\usetikzlibrary{decorations.pathmorphing}
\usetikzlibrary[patterns]
%\tikzstyle{every text node part}
%\usetikzlibrary{arrows,backgrounds,positioning,fit} 
\usetikzlibrary{calc}
% para gerar graficos no latex
\usepackage{pgfplots}
\pgfplotsset{compat=newest}

\usepackage[T1]{fontenc}
\usepackage{amsfonts}
\usepackage{amssymb}
\usepackage{amsmath}
\usepackage{MnSymbol}
\usepackage{amsmath,amsthm}
\usepackage[spanish]{babel}
\usepackage[utf8]{inputenc}
\usepackage{listings}
\usepackage{multirow, bigdelim}
\usepackage{hhline}
\usepackage{calc}
\usepackage{backnaur}
\usepackage{pbox}



% \usepackage{algpseudocode}
% \usepackage{algorithmicx}
\usepackage[Algoritmo]{algorithm}
\usepackage[noend]{algorithmic}
\usepackage{float}
\usepackage[style=numeric,sorting=none,backend=biber]{biblatex}
%%%%
\usepackage{ragged2e}
\usepackage{chronosys}

\newcommand{\foreign}[1]{{\it #1}}
\renewcommand{\algorithmicrequire}{\textbf{Entrada:}}
\renewcommand{\algorithmicensure}{\textbf{Salida:}}
\renewcommand{\algorithmicif}{\textbf{si}}
\renewcommand{\algorithmicend}{\textbf{fin}}
\renewcommand{\algorithmicfor}{\textbf{por}}
\renewcommand{\algorithmicdo}{\textbf{hacer}}
\renewcommand{\algorithmicthen}{\textbf{entonces}}
\renewcommand{\algorithmicreturn}{\textbf{retorna}}


\setbeamertemplate{bibliography entry title}{}
\setbeamertemplate{bibliography entry location}{}
\setbeamertemplate{bibliography entry note}{}

\newcounter{saveenumi}
\newcommand{\seti}{\setcounter{saveenumi}{\value{enumi}}}
\newcommand{\conti}{\setcounter{enumi}{\value{saveenumi}}}

%\usepackage{shadethm}

%\definecolor{shadethmcolor}{rgb}{.75,.75,.75}

%\newshadetheorem{theorem}{\scshape Teorema}[chapter]
%\newtheorem{teorema}[theorem]{\scshape Teorema}
%\newtheorem{proposicao}[theorem]{\scshape Proposição}
%\newtheorem{corolario}[theorem]{\scshape Corolário}
%\newtheorem{lema}[theorem]{\scshape Lema}
%\newtheorem{definicao}[theorem]{\scshape Definição}
%\newtheorem{conjectura}[theorem]{\scshape Conjectura}
%\newtheorem{escolio}[theorem]{\scshape Escólio}
%\newtheorem{exemplo}[theorem]{\scshape Exemplo}
%\newtheorem{exemplos}[theorem]{\scshape Exemplos}
%\newtheorem{propriedade}[theorem]{\scshape Propriedade}

\renewcommand{\u}{{\bf u}}
\renewcommand{\v}{{\bf v}}
\renewcommand{\sin}{\operatorname{sen}}
\renewcommand{\tan}{\operatorname{tg}}
\providecommand{\cas}{\operatorname{cas}}
\providecommand{\mdc}{\mathrm{mdc}}
\providecommand{\f}{{\bf f}}

\newcommand{\ie}{\textit{i.e.}}
\newcommand{\eg}{\textit{e.g.}}
%\newcommand{\qed}{\hfill $\square$}

\renewcommand\Re{\operatorname{Re}}
\renewcommand\Im{\operatorname{Im}}

\providecommand{\x}{{\bf x}}
\providecommand{\y}{{\bf y}}
\providecommand{\w}{{\bf w}}
\providecommand{\f}{{\bf f}}
\providecommand{\q}{{\bf q}}
\providecommand{\bfa}{{\bf a}}
\providecommand{\bfb}{{\bf b}}
\providecommand{\bfc}{{\bf c}}
\providecommand{\bfd}{{\bf d}}
\providecommand{\bfe}{{\bf e}}
\providecommand{\bfs}{{\bf s}}
\providecommand{\bfz}{{\bf z}}
\providecommand{\zero}{{\bf 0}}
\providecommand{\spn}{\mathrm{span}}
\providecommand{\posto}{\mathrm{posto}}
\providecommand{\nul}{\mathrm{nul}}
\providecommand{\proj}{\mathrm{proj}}
\providecommand{\tr}{\mathrm{tr}}
\providecommand{\sgn}{\mathrm{sgn}}

\providecommand{\cov}{\mathrm{cov}}

\providecommand{\dilation}{\mathcal{D}}
\providecommand{\erosion}{\mathcal{E}}
\providecommand{\open}{\mathcal{O}}
\providecommand{\close}{\mathcal{C}}

\newcommand*{\Bhat}{\skew{3}{\hat}{B}}

\mode<presentation>
{
  \setbeamertemplate{background canvas}[vertical shading][bottom=white!10,top=blue!10]
%  \usetheme{Berkeley}
%  \usetheme{CambridgeUS}
%  \usetheme{Madrid}
  \usetheme{Warsaw}
  \usefonttheme[onlysmall]{structurebold}
  
  \setbeamertemplate{headline}{}
  
%   \setbeamercovered{invisible} % default
  \setbeamercovered{ transparent, again covered={\opaqueness{25}} } % =15%
%   \setbeamercovered{transparent=50}
%   \setbeamercovered{dynamic}

%   \setbeamercovered{again covered={\opaqueness<1->{25}}}
}

% copiado do site:
% http://latex-beamer-class.10966.n7.nabble.com/Covering-images-transparent-i-e-dimmed-figures-td1504
% . html
\usepackage{ifthen}

\makeatletter
\newcommand{\includecoveredgraphics}[2][]{
    \ifthenelse{\the\beamer@coveringdepth=1}{
        \tikz
            \node[inner sep=0pt,outer sep=0pt,opacity=0.15]
                {\includegraphics[#1]{#2}};
    }{
        \tikz
            \node[inner sep=0pt,outer sep=0pt]
                {\includegraphics[#1]{#2}};%
    }
} 

\newcommand{\thickhline}{%
    \noalign {\ifnum 0=`}\fi \hrule height 1pt
    \futurelet \reserved@a \@xhline
}

\makeatother

%\pgfdeclareimage[height=1.4cm]{logo_XIVsm}{semanauniversitaria}

%% put XIVsm logo in bottom left
%\setbeamertemplate{sidebar left}{
%%   \vfill%
 %  \rlap{\hskip0.0cm
  %       %\href{http://www.uece.br/semanauniversitaria}
   %      {\pgfuseimage{logo_XIVsm}}}
   %%\vskip2pt%
   %%\llap{\usebeamertemplate***{navigation symbols}\hskip0.1cm}%
   %%\vskip2pt%
%}

\DeclareMathOperator*{\argmax}{arg\,max}

% para que entre bien el título en el footer
\setbeamertemplate{footline}
{
  \leavevmode%
  \hbox{%
  \begin{beamercolorbox}[wd=.3\paperwidth,ht=2.25ex,dp=1ex,center]{author in head/foot}%
    \usebeamerfont{author in head/foot}\insertshortauthor
  \end{beamercolorbox}%
  \begin{beamercolorbox}[wd=.7\paperwidth,ht=2.25ex,dp=1ex,center]{title in head/foot}%
    \usebeamerfont{title in head/foot}\insertshorttitle\hspace*{3em}
    \insertframenumber{} / \inserttotalframenumber\hspace*{1ex}
  \end{beamercolorbox}}%
  \vskip0pt%
}

\setbeamertemplate{navigation symbols}{}

\title{Dise\~no de Interfaces de Usuario basado en Reconocimiento del Habla}
\author{Rodrigo Parra y Jorge Ram\'{i}rez}

\date{San Lorenzo, 2014}


\addbibresource{referencias.bib}
\begin{document}


\frame{\titlepage}

\frame{\tableofcontents}

\justifying
\section{Introducci\'on}

\begin{frame}{Introducci\'on}

\visible<2>{\centering{¿Qu\'e es el reconocimiento del habla?}}
\begin{figure}[H]
\centering
\includegraphics[width=0.7\linewidth]{./graphics/speech_intro.jpg}
\end{figure}


\end{frame}
\section*{Antecedentes del Reconocimiento del Habla}

\begin{frame}{Antecedentes del Reconocimiento del Habla}
    % \begin{itemize}
    %     \item 50s: Inicios del &aacute;rea de investigaci&oacute;n.
    %     \item 60s: Desarrollo del \emph{Dynamic Time Warping}.
    %     \item 70s: Publicaci&oacute;n de la teor&iacute;a de los modelos ocultos de Markov (HMM).
    %     \item 80s: Adopci&oacute;n del modelo estad&iacute;stico. %Mencionar N-gramas!
    %     \item $&gt;$ 90s: Llegada del reconocimiento del habla al usuario com&uacute;n.
    %     \item 2016-2019: Meseta de productividad seg&uacute;n Gartner.
    % \end{itemize}

    % \begin{chronology}[5]{1950}{2019}{\textwidth}
    %     \event{1950}{Inicios del \'area de investigaci\'on.}
    %     \event{1960}{Desarrollo del \emph{Dynamic Time Warping}.}
    %     \event{1970}{Publicaci\'on de la teor{\'\i}a de los modelos ocultos de Markov (HMM).}
    %     \event{1980}{Adopci\'on del modelo estad{\'\i}stico.}
    %     \event[1990]{2014}{Llegada del reconocimiento del habla al usuario com\'un.}
    %     \event[2016]{2019}{Meseta de productividad seg\'un Gartner.}
    % \end{chronology}

\startchronology[startyear=1950,stopyear=2020,color=blue, startdate=false, stopdate=false]
  \uncover<8-8>{\chronoperiode[textdepth=20pt, startdate=false, stopdate=false, colorbox=lightgray, color=cyan]{2016}{2020}{\emph{Gartner}}}
  % \chronoperiode{1990}{2014}{Llegada del reconocimiento del habla al usuario com\'un.}
  \uncover<7-8>{\chronoevent[markdepth=100pt, colorbox=lightgray, year=false, date=false, barre=false]{2008}{\emph{Google Voice Search}}}
  \chronoperiode[startdate=false, stopdate=false, color=blue, ifcolorbox=false]{2007}{2008}{}  
  \uncover<6-8>{\chronoevent[markdepth=60pt, colorbox=lightgray, year=false, date=false, barre=false]{2007}{\emph{Apple Siri}}}
  \chronoperiode[startdate=false, color=blue, ifcolorbox=false]{1997}{2007}{}  
  \uncover<4-8>{\chronoevent[markdepth=130pt, colorbox=lightgray, year=false, date=false, barre=false]{1986}{N-Gramas}}
  \chronoperiode[startdate=false, color=blue, ifcolorbox=false]{1986}{1997}{}  
  \uncover<5-8>{\chronoevent[markdepth=30pt, colorbox=lightgray, year=false, date=false, barre=false]{1997}{\emph{Dragon Naturally Speaking}}}
  \chronoperiode[startdate=false, color=blue, ifcolorbox=false]{1975}{1986}{}  
  \uncover<3-8>{\chronoevent[markdepth=90pt, colorbox=lightgray, year=false, date=false, barre=false]{1975}{Modelos Ocultos de Markov}}
  \chronoperiode[startdate=false, color=blue, ifcolorbox=false]{1968}{1975}{}
  \uncover<2-8>{\chronoevent[markdepth=50pt, colorbox=lightgray, year=false, date=false, barre=false]{1968}{\emph{Dynamic Time Warping}}}
  \chronoperiode[startdate=false, color=blue, ifcolorbox=false]{1952}{1968}{}
  \uncover<1-8>{\chronoevent[colorbox=lightgray, year=false, date=false, barre=false]{1952}{\qquad Inicia investigaci\'on}}
  \chronoperiode[startdate=false, color=blue, ifcolorbox=false]{1950}{1952}{}
\stopchronology

% \startchronology
% \chronoevent[markdepth=60pt]{1525}{Label B}
% \chronoevent{1500}{Label A}
% \stopchronology

\end{frame}
\section{Actualidad y Aplicaciones}

\begin{frame}{Actualidad y Aplicaciones}
El panorama actual del reconocimiento del habla se caracteriza por su aplicaci\'on
en diversos \'ambitos, algunos de los cuales se mencionan a continuaci\'on:

\begin{itemize}
	\item{\textbf{Medicina y Derecho:}}
    existen aplicaciones comerciales y trabajos de investigaci\'on enfocados a la transcripci\'on de registros 
    m\'edicos \cite{LaiMedSpeak1997, USATodayHospitals} y reportes legales \cite{FalavignaAutomatic2009, NuanceLegal}
	mediante reconocimiento del habla.

	\item{\textbf{Milicia:}}
	se destacan los trabajos de Beek \cite{BeekAn1977} y Weinstein \cite{WeinsteinOpportunities1991},
	en los cuales se clasifican las potenciales aplicaciones de tecnolog\'ias del habla en 
	cinco categor{\'\i}as: seguridad, mando y control, transmisi\'on de datos y comunicaci\'on, 
	procesamiento de voz distorsionada y aplicaciones para entrenamiento.
\end{itemize}
\end{frame}
\begin{frame}{Actualidad y Aplicaciones (2)}
\begin{itemize}
	\item{\textbf{Telefon\'ia:}}
	las aplicaciones de reconocimiento del habla buscan reducir los costos, como
	la automatizaci\'on de consulta de directorio, 
	o producir ganancias mediante nuevos modelos de servicio, como servicios bancarios
	y de servicios de reserva \cite{RabinerApplications1997}.

	\item{\textbf{Accesibilidad:}}
	las interfaces mediante voz poseen un gran potencial para usuarios con alg\'un tipo de discapacidad
	que les impida manejar apropiadamente el rat\'on y/o el teclado o visualizar la informaci\'on en el monitor.
	Varios ejemplos de este tipo de aplicaciones se mencionan en \cite{AnanthiSurvey2013}.
\end{itemize}
\end{frame}
\begin{frame}{Actualidad y Aplicaciones (3)}
\begin{itemize}
	\item{\textbf{Industria Automotriz:}}
	seg\'un \cite{Kirriemuir2003Speech}, el desarrollo de sistemas de reconocimiento del habla para autom\'oviles
	se divide en las siguientes \'areas: dispositivos manos libres, control
	de los dispositivos de navegaci\'on, interacci\'on con el sistema de control y sistemas de manejo por voz.

	\item{\textbf{Tel\'efonos Inteligentes e Interfaces Web:}}
	se destaca el surgimiento de los asistentes virtuales,
	como \emph{Google Now} \cite{GoogleNow} y \emph{Siri} \cite{AppleSiri}. Adem\'as,
	puede mencionarse la \emph{Web Speech API} \cite{GoogleWebSpeechAPI}, una propuesta de \emph{Google}
	a la \emph{W3C} que permite integrar reconocimiento y s\'intesis del habla a las aplicaciones web.
\end{itemize}
\end{frame}
\begin{frame}{Actualidad y Aplicaciones (4)}
\begin{itemize}
	\item{\textbf{Videojuegos y Dom\'otica:}}
	existen juegos comerciales que interactuan con el usuario
	mediante la voz, como \emph{Say-N-Play}\cite{SayNPlay}. En el caso de la dom\'otica, 
	est\'an disponibles productos que permiten controlar los artefactos del hogar mediante la voz, 
	como \emph{HAL} \cite{HAL}.
\end{itemize}
\end{frame}

\section{Proceso B\'asico del Reconocimiento del Habla}

\begin{frame}{Proceso B\'asico del Reconocimiento del Habla}

\begin{quote}
\emph{La b\'usqueda de la oraci\'on m\'as probable W perteneciente al lenguaje L, dada la entrada ac\'ustica 0.}
\end{quote}

\begin{align}
\hat{W} = \argmax_{W \in L} \overbrace{P(O|W)}^\text{M. ac\'ustico}\overbrace{P(W)}^\text{M. de lenguaje}
\label{eq:asrFundamental}
\end{align}
\end{frame}

\begin{frame}{Proceso B\'asico del Reconocimiento del Habla (2)}

\begin{figure}[H] 
\centering
\includegraphics[width=0.8\textwidth]{./graphics/proceso.png}
\caption{Proceso del reconocimiento del habla. Traducido a partir de \protect\cite{VerenichASR}.}
\label{figure:proceso}
\end{figure}
\end{frame}

\begin{frame}{Proceso B\'asico del Reconocimiento del Habla (3)}
\framesubtitle{Fase 1: Extracci\'on de caracter{\'\i}sticas}
\begin{figure}[H]
\centering
\includegraphics[width=0.4\linewidth]{./graphics/formants.png}
\caption{Representaci\'on del espectro en el cual pueden identificarse los picos espectrales o formantes 
\cite{Jurafsky}.}
\label{figure:formants}
\end{figure}


\begin{figure}[H]
\centering
\includegraphics[width=0.7\linewidth]{./graphics/spectrogram.png}
\caption{Representaci\'on de un espectrograma, puede verse como una colecci\'on de espectros  ubicados uno despu\'es de otro \cite{Jurafsky}.}
\label{figure:spectrogram}
\end{figure}

\end{frame}

\begin{frame}{Proceso B\'asico del Reconocimiento del Habla (4)}
\framesubtitle{Fase 1: Extracci\'on de caracter{\'\i}sticas}

\begin{figure}[H] 
\centering
\includegraphics[width=0.8\textwidth]{./graphics/extraccion.png}
\caption{Fase de extracci\'on de caracter{\'\i}sticas. Gr\'afico basado en \cite{VerenichASR}.}
\label{figure:hmm}
\end{figure}
\end{frame}


\begin{frame}{Proceso B\'asico del Reconocimiento del Habla (5)}
\framesubtitle{Fase 2: Decodificaci\'on: Modelo de Lenguaje}
\uncover<1-3>{Probabilidad de ocurrencia de una secuencia de palabras $x_1,x_2,\ldots,x_n$ para un lenguaje dado.}
\vspace*{2\baselineskip}
\begin{itemize}
    \uncover<2-3>{\item Basado en N-Gramas
        \small \begin{equation*}
            \hspace*{-1.5cm} 
            p(\text{el, hombre, corre}) = p(el \mid \text{\textless} s\text{\textgreater}) \, 
            p(\text{\emph{hombre}} \mid el) \, p(corre \mid \text{\emph{hombre}}) \, 
            p(\text{\textless} /s\text{\textgreater} \mid corre)
        \end{equation*}}
        \vspace*{1\baselineskip}
        \uncover<3-3>{\item Basado en Gramática
        \begin{bnf*}
            \bnfprod{pregunta}
            {\bnfts{Cu\'al} \bnfsp \bnfts{es} \bnfsp \bnfts{la}  \bnfpn{info} \bnfsp  \bnfts{en} \bnfsp \bnfpn{ciudad}} \\
            \bnfprod{info}
            {\bnfts{temperatura} \bnfor \bnfts{presi\'on atmosf\'erica} \bnfor \bnfts{hora}} \\
            \bnfprod{ciudad}
            {\bnfts{Par{\'\i}s} \bnfor \bnfts{Nueva York} \bnfor \bnfts{Roma}}
        \end{bnf*}}
    \end{itemize}
\end{frame}

\begin{frame}{Proceso B\'asico del Reconocimiento del Habla (6)}
\framesubtitle{Fase 2: Decodificaci\'on - Modelo Acústico}

\begin{itemize}
    \item Modelos Ocultos de Markov
    \item Diccionario Fonético
\end{itemize}

\begin{figure}[H] 
\centering
\includegraphics[width=0.5\textwidth]{./graphics/hmm_palabra.png}
\caption{Representaci\'on ac\'ustica de la palabra ``mes''. Basado en \cite{Jurafsky}.}
\label{figure:hmm-palabra}
\end{figure}

\end{frame}

\begin{frame}{Proceso B\'asico del Reconocimiento del Habla (7)}
\framesubtitle{Fase 2: Decodificaci\'on - Búsqueda}

\begin{itemize}
    \item Algoritmo de Viterbi.
    \item Algoritmo A*.
\end{itemize}

\begin{figure}[H] 
\centering
\includegraphics[width=0.5\textwidth]{./graphics/espacio.png}
\caption{Espacio de b\'usqueda para un lenguaje simple de cuatro palabras. Traducido a partir de \cite{RenalsSearch}.}
\label{figure:espacio-busqueda}
\end{figure}

\end{frame}


\begin{frame}{Proceso B\'asico del Reconocimiento del Habla (8)}
\framesubtitle{Fase 2: Decodificaci\'on}
\begin{figure}[H] 
\centering
\includegraphics[width=0.8\textwidth]{./graphics/decodificacion.png}
\caption{Fase de decodificaci\'on. Gr\'afico basado en \cite{VerenichASR}.}
\label{figure:decoding}
\end{figure}
\end{frame}

\section{Tecnolog\'ias y Herramientas}

\begin{frame}{Tecnolog\'ias y Herramientas}
\setbeamercovered{transparent}
\begin{columns}
\column{0.30\linewidth}
\only<1-3>{\begin{itemize}
    \uncover<+->{\item Clasificaci\'on}
    \uncover<+->{\item Criterios}
\end{itemize}}
\column{0.7\linewidth}
\only<3>{\begin{figure}[H]
\centering
\includegraphics[width=1.1\linewidth]{./graphics/esquema-herramientas.png}
\caption{Esquema de la clasificaci\'on de tecnolog{\'\i}as y herramientas}
\label{figure:esquema-herramientas}
\end{figure}}
\end{columns}

\only<4>{\begin{figure}[H]
    \includegraphics[width=1\linewidth]{./graphics/esquema-herramientas-2.png}
    \caption{Esquema de la clasificaci\'on de tecnolog{\'\i}as y herramientas}
\end{figure}}
\end{frame}
\section{Definici\'on del Problema}

\begin{frame}{Definici\'on del Problema}
\end{frame}

\section{Soluci\'on Propuesta}

\begin{frame}{Soluci\'on Propuesta (1/5)}
\framesubtitle{TamTam Edit}

\begin{figure}[H]
\centering
\includegraphics[width=0.8\textwidth]{./graphics/ui-tamtam-edit.png}
\caption{Interfaz de \emph{Tamtam Edit} y sus secciones principales.}
\label{figure:ui-tamtam}
\end{figure}

\end{frame}

\begin{frame}{Soluci\'on Propuesta (2/5)}
\framesubtitle{TamTam Listens}

\only<1>{\begin{figure}[H] 
\centering
\includegraphics[width=1.0\textwidth]{./graphics/tamtam-proceso-4.png}
\caption{Arquitectura de Tamtam Listens}
\label{figure:tamtam-listens-proceso-4}
\end{figure}}

\only<2>{\begin{figure}[H] 
\centering
\includegraphics[width=1.0\textwidth]{./graphics/tamtam-proceso-1.png}
\caption{Arquitectura de Tamtam Listens. Motor de Reconocimiento}
\label{figure:tamtam-listens-proceso-1}
\end{figure}}

\only<3>{\begin{figure}[H] 
\centering
\includegraphics[width=1.0\textwidth]{./graphics/tamtam-proceso-3.png}
\caption{Arquitectura de Tamtam Listens. Servicio de reconocimiento}
\label{figure:tamtam-listens-proceso-3}
\end{figure}}

\only<4>{\begin{figure}[H] 
\centering
\includegraphics[width=1.0\textwidth]{./graphics/tamtam-proceso-4.png}
\caption{Arquitectura de Tamtam Listens}
\label{figure:tamtam-listens-proceso-4}
\end{figure}}

\end{frame}

\begin{frame}{Soluci\'on Propuesta (3/5)}
\framesubtitle{TamTam Listens}

\only<1>{\begin{figure}[H] 
\centering
\includegraphics[width=1.0\textwidth]{./graphics/interaccion1.png}
\caption{Ejemplo de interacci\'on}
\label{figure:tamtam-listens-interaccion}
\end{figure}}

\only<2>{\begin{figure}[H] 
\centering
\includegraphics[width=1.0\textwidth]{./graphics/interaccion2.png}
\caption{Ejemplo de interacci\'on}
\label{figure:tamtam-listens-interaccion}
\end{figure}}

\only<3>{\begin{figure}[H] 
\centering
\includegraphics[width=1.0\textwidth]{./graphics/interaccion3.png}
\caption{Ejemplo de interacci\'on}
\label{figure:tamtam-listens-interaccion}
\end{figure}}

\only<4>{\begin{figure}[H] 
\centering
\includegraphics[width=1.0\textwidth]{./graphics/interaccion4.png}
\caption{Ejemplo de interacci\'on}
\label{figure:tamtam-listens-interaccion}
\end{figure}}

\only<5>{\begin{figure}[H] 
\centering
\includegraphics[width=1.0\textwidth]{./graphics/interaccion5.png}
\caption{Ejemplo de interacci\'on}
\label{figure:tamtam-listens-interaccion}
\end{figure}}


\only<6>{\begin{figure}[H] 
\centering
\includegraphics[width=1.0\textwidth]{./graphics/interaccion6.png}
\caption{Ejemplo de interacci\'on}
\label{figure:tamtam-listens-interaccion}
\end{figure}}

\only<7>{\begin{figure}[H] 
\centering
\includegraphics[width=1.0\textwidth]{./graphics/interaccion7.png}
\caption{Ejemplo de interacci\'on}
\label{figure:tamtam-listens-interaccion}
\end{figure}}

\only<8>{\begin{figure}[H] 
\centering
\includegraphics[width=1.0\textwidth]{./graphics/interaccion8.png}
\caption{Ejemplo de interacci\'on}
\label{figure:tamtam-listens-interaccion}
\end{figure}}

\only<9>{\begin{figure}[H] 
\centering
\includegraphics[width=1.0\textwidth]{./graphics/interaccion9.png}
\caption{Ejemplo de interacci\'on}
\label{figure:tamtam-listens-interaccion}
\end{figure}}

\only<10>{\begin{figure}[H] 
\centering
\includegraphics[width=1.0\textwidth]{./graphics/interaccion10.png}
\caption{Ejemplo de interacci\'on}
\label{figure:tamtam-listens-interaccion}
\end{figure}}

\only<11>{\begin{figure}[H] 
\centering
\includegraphics[width=1.0\textwidth]{./graphics/interaccion11.png}
\caption{Ejemplo de interacci\'on}
\label{figure:tamtam-listens-interaccion}
\end{figure}}

\only<12>{\begin{figure}[H] 
\centering
\includegraphics[width=1.0\textwidth]{./graphics/interaccion12.png}
\caption{Ejemplo de interacci\'on}
\label{figure:tamtam-listens-interaccion}
\end{figure}}

\only<13>{\begin{figure}[H] 
\centering
\includegraphics[width=1.0\textwidth]{./graphics/interaccion13.png}
\caption{Ejemplo de interacci\'on}
\label{figure:tamtam-listens-interaccion}
\end{figure}}

\end{frame}


\begin{frame}[fragile]{Soluci\'on Propuesta (4/5)}
\framesubtitle{Modelo de Lenguaje}
\begin{figure}[H]
\lstset{basicstyle=\ttfamily\scriptsize}
\begin{lstlisting}
#JSGF V1.0;
grammar tamtam;

public <tamtam-listens> = <comando> | <pagina> | <pista-a>  |
                          <pista-b>  | <seleccionar-compas> | 
                          <crear-nota> | <seleccionar-nota> | 
                          <duplicar-nota> | <borrar-nota>   | 
                          <volumen> | <tempo> | <loop>      |
                          <configurar-nota> ;

<comando>  = REPRODUCIR MUSICA  | PAUSAR MUSICA   |
             PARAR MUSICA | GENERAR MUSICA        | 
             CREAR NUEVA MUSICA | EXPORTAR MUSICA | 
             SALIR DE TAMTAM;
<pagina>   = ( CREAR NUEVA | DUPLICAR | LIMPIAR ) 
              PARTITURA | PARTITURA <orden>;
<orden>    = ( ANTERIOR | SIGUIENTE );
<loop>     = (COMODIN)+;
\end{lstlisting}
\caption{Fragmento de la gram\'atica utilizada en \emph{Tamtam Listens}.}
\label{figure:fragmento-gram}
\end{figure} 
\end{frame}

\begin{frame}[fragile]{Soluci\'on Propuesta (5/5)}
\framesubtitle{Modelo Acústico}
\begin{itemize}
    \item Proyecto Voxforge
    \item Diccionario Fonético    
\end{itemize}

    \begin{figure}[H]
    \begin{lstlisting}
    REPRODUCIR RR E P R O D U S I R
    PAUSAR P A U S A R
    PARAR  P A R A R
    GENERAR  J E N E R A R
    PARTITURA P A R T I T U R A
    SIGUIENTE S I G I E N T E
    ANTERIOR A N T E R I O R
    \end{lstlisting}
    \caption{Fragmento del diccionario fon\'etico utilizado en \emph{Tamtam Listens}.}
    \label{figure:fragmento-dic}
\end{figure}

\end{frame}


\section{Evaluaci\'on}

\begin{frame}{Evaluaci\'on}
\framesubtitle{Metodolog\'ia}

\begin{itemize}
    \item 12 estudiantes.
    \item Laboratorio de Inform\'atica de la FPUNA.
    \item 2 semanas.
\end{itemize}

Las características de cada sesión son:
\begin{itemize}
    \item Duración: aproximadamente una hora
    \item Participantes:
        \begin{itemize}
            \item Usuario
            \item Facilitador
            \item Observador
        \end{itemize}
    \item Fases:
        \begin{itemize}
            \item Test de Memoria
            \item Entrenamiento
            \item Tareas
            \item Encuesta de Usabilidad
            \item Recopilación y Análisis de Datos
        \end{itemize}   
\end{itemize}
\end{frame}


\begin{frame}{Evaluaci\'on (2)}
\framesubtitle{Análisis de Datos}
    \begin{itemize}
        \item Memoria del Usuario: Test de Memoria ($M$)
        \item Correctitud de la Aplicación
            \begin{itemize}
                \item Tasa de Acierto ($A$)
                \item Tasa de Error de Comandos ($E_1$)
            \end{itemize}
        \item Error Humano
            \begin{itemize}
                \item Tasa de Error Humano ($E_2$)
                    \begin{itemize}
                        \item Tasa de Error por Longitud del Comando
                        \item Tasa de Error por Nivel Contextual del Comando
                        \item Tasa de Error por Comando
                        \item Distribuci\'on del Error Humano por Etapas de la Sesi\'on 
                    \end{itemize}
                \item Cantidad de Errores ($E_3$)
            \end{itemize}
        \item Eficiencia
            \begin{itemize}
                \item Duración de Tareas Uno y Dos ($T_{1+2}$)
                \item Duración de Tareas Tres y Cuatro($T_{3+4}$)
                \item Cantidad de Comandos Utilizados ($U$)
                \item Correctitud de la Tarea Cuatro ($C$)
            \end{itemize}
        \item Satisfacción del Usuario: Encuesta de Usabilidad
    \end{itemize}
\end{frame}
\section{Resultados Obtenidos}

\begin{frame}{Resultados Obtenidos (1/7)}
\framesubtitle{Metodolog\'ia}

\framesubtitle{An\'alisis de Datos}
\vspace{-0.5em}
\begin{table}[H]
\centering
\scriptsize
\begin{tabular}{|p{0.9cm}|p{4.2cm}|p{2cm}|p{2cm}}
\hhline{---~}
M\'etrica  &   Descripci\'on & Valor  \\
\hhline{---~}
$M$ &   Resultado del test de memoria &  \textcolor{ForestGreen}{12,75 palabras} & \rdelim\}{1}{2cm}[Memoria del Usuario] \\
\rule{0pt}{4ex}$A$       &   Tasa de Acierto               &  \textcolor{ForestGreen}{83,70 \%} & \rdelim\}{3}{2cm}[\parbox{3cm-\tabcolsep-\widthof{$\Bigg]$}}{Correctitud de la Aplicaci\'on}] \\
 \rule{0pt}{4ex}$E_1$     & Tasa de Error de Comandos     &   \textcolor{Red}{16,30 \%}  \\
 \rule{0pt}{4ex}$E_2$     & Tasa de Error Humano          &   \textcolor{Red}{5,91 \%} &  \rdelim\}{3}{2cm}[Error Humano] \\
 \rule{0pt}{4ex}$E_3$     & Cantidad de Errores           &   \textcolor{Red}{11,83 errores}  \\
 \rule{0pt}{4ex}$T_{1+2}$ & Duraci\'on de Tareas Uno y Dos  & \textcolor{Red}{13,83 minutos}  & \rdelim\}{6}{2cm}[Eficiencia] \\
 \rule{0pt}{4ex}$T_{3+4}$ & Duraci\'on de Tareas Tres y Cuatro & \textcolor{Red}{18,35 minutos} \\
 \rule{0pt}{4ex}$C$       & Correctitud de la Tarea Cuatro  &    \textcolor{ForestGreen}{87,5 \%}  \\
 \rule{0pt}{4ex}$U$       & Cantidad de Comandos Utilizados &    \textcolor{ForestGreen}{40.67 comandos} \\
\cline{1-3} 
\end{tabular}
\end{table}

\textcolor{ForestGreen}{Mayor es mejor}

\textcolor{Red}{Menor es mejor}
\end{frame}

\begin{frame}{Resulados Obtenidos (2/7)}
\framesubtitle{Correlaci\'on}
\only<1>{\centering{Tasa de Acierto y Medidas del Error Humano}}
\only<2>{\begin{figure}[ht]
\centering
\includegraphics[width=1\linewidth]{./graphics/correlacion1.png}
\end{figure}}

\only<3>{Comandos utilizados, Medidas de Error Humano y Tasa de Acierto}
\only<4>{\begin{figure}[ht]
\centering
\includegraphics[width=1\linewidth]{./graphics/correlacion2.png}
\end{figure}}

\only<5>{Memoria del usuario, Tasa de Error Humano y Duraci\'on de las Tareas}
\only<6>{\begin{figure}[ht]
\centering
\includegraphics[width=0.9\linewidth]{./graphics/correlacion3.png}
\end{figure}}
\end{frame}


\begin{frame}{Resulados Obtenidos (3/7)}
\framesubtitle{Correlaci\'on}
\only<1>{\centering{Tasa de Acierto y Correctitud de la Tarea 4}}
\only<2>{\begin{figure}[ht]
\centering
\includegraphics[width=0.9\linewidth]{./graphics/correlacion4.png}
\end{figure}}

\only<3>{\centering{Medidas de Error y Duraci\'on de las Tareas}}
\only<4>{\begin{figure}[ht]
\centering
\includegraphics[width=0.9\linewidth]{./graphics/correlacion5.png}
\end{figure}}
\end{frame}

\begin{frame}{Resultados Obtenidos (4/7)}
\framesubtitle{An\'alisis del Error Humano}

\only<1>{\begin{figure}[ht]
\centering
\includegraphics[width=0.9\linewidth]{./graphics/longitud-error.png}
\caption{Tasa de error humano por longitud del comando.}
\end{figure}}

\only<2>{\begin{table}[H]
\centering
\footnotesize
\begin{tabular}{|p{1.6cm}|p{1.6cm}|p{1.6cm}|p{1.6cm}|}
\hline
    Contexto & General & Pista & Comp\'as \\
\hline
Promedio & 13,25 & 3,57 & 5,16 \\
\hline
\end{tabular}
\caption{Tasa de error humano por nivel contextual del comando.}
\label{sec:error-contexto}
\end{table}}

\only<3>{\begin{table}[H]
\centering
\footnotesize
\begin{tabular}{|l|p{3cm}|}
\hline
Comando & Tasa de Error \\
\hline
crear nueva partitura & 18,38 \\
duplicar pista uno en pista dos & 17,5 \\
duplicar pista uno en pista tres & 16,67 \\
duplicar pista tres en pista cuatro & 13,25 \\
comp\'as cuatro & 13,19 \\
\hline
\end{tabular}
\caption{Lista de comandos con mayor tasa de error humano promedio.}
\label{sec:tabla-lista-comandos-error}
\end{table}}
\end{frame}


\begin{frame}{Resultados Obtenidos (5/7)}
\framesubtitle{An\'alisis del Error Humano}

\begin{columns}
\column{0.3\linewidth}
\centering
\begin{table}
\tiny
\begin{tabular}{|c|c|}
\hline
    Etapa & \% de la Tasa \\ & de Error Total \\
    \hline
0-10  &  11,62 \\
10-20 &  13.49 \\
20-30 &  12,29 \\
30-40 &  17,78 \\
40-50 &  6,85 \\
50-60 &  3,75 \\
60-70 &  9,92 \\
70-80 &  2,76 \\
80-90 &  12,12 \\
90-100 & 9,42 \\
    \hline
\end{tabular}
\caption{Distribuci\'on del error humano por etapas de la sesi\'on.}
\label{sec:error-tiempo}
\end{table}
\column{0.7\linewidth}
\begin{figure}
\centering
\includegraphics[width=1\linewidth]{./graphics/error_tiempo.png}
\caption{Distribuci\'on del error humano por etapas de la sesi\'on.}
\label{figure:gerror-tiempo}
\end{figure}
\end{columns}

\end{frame}

\begin{frame}{Resultados Obtenidos (6/7)}
\framesubtitle{Encuesta}
\begin{columns}
\column{0.35\linewidth}
\begin{table}[H] 
\centering
\tiny
\begin{tabular}{|r|r|r|r|r|}
\hline
            & Promedio \\
\hline
Vocabulario    & 6.17 \\
Comandos    & 6.58 \\
Entrenamiento  & 6.25 \\
Interfaz por Voz & 5.83 \\
\hline
\end{tabular}
\caption{Resumen de la encuesta realizada.}
\label{sec:tabla-encuesta}
\end{table} 
\column{0.65\linewidth}
\begin{figure}[ht]
\centering
\includegraphics[width=1\linewidth]{./graphics/kiviat0.png}
\caption{Gr\'afico radial resumen de la encuesta realizada.}
\label{figure:kiviat-encuesta1}
\end{figure}
\end{columns}
\end{frame}

\begin{frame}{Resultados Obtenidos (7/7)}
\framesubtitle{Encuesta}

\begin{columns}
\column{0.35\linewidth}
\begin{equation*}
s_a=\frac{s-min_i}{max_i-min_i}
\end{equation*}

\begin{table}[H] 
\centering
\tiny
\begin{tabular}{|r|r|r|r|r|}
\hline
            & Promedio \\
\hline
Vocabulario    & 0.45 \\
Comandos    & 0.86 \\
Entrenamiento  & 0.55 \\
Interfaz por Voz & 0.5 \\
\hline
\end{tabular}
\caption{Resumen de la encuesta realizada. Valores reescalados.}
\label{sec:tabla-encuesta-normalizada}
\end{table}
\column{0.65\linewidth}
\begin{figure}[ht]
\centering
\includegraphics[width=1\linewidth]{./graphics/kiviat.png}
\caption{Gr\'afico radial resumen de la encuesta realizada. Valores reescalados.}
\label{figure:kiviat-encuesta2}
\end{figure}
\end{columns}
\end{frame}
\section{Conclusiones}

\begin{frame}{Dise\~no de la Interfaz}

\begin{itemize}
    \item La naturalidad del lenguaje es de gran importancia para la interfaz.
    \item Interactuar con la aplicaci\'on, no con la interfaz gr\'afica.
    \item Utilizar el sonido como medio de retroalimentaci\'on.
\end{itemize}

\end{frame}

\begin{frame}{Implementaci\'on de la Interfaz}

\begin{itemize}
    \item Seleccionar las herramientas de acuerdo al \mbox{proyecto}.
    \item Considerar la posibilidad de errores en el \emph{software}.
    \item Realizar pruebas y modificaciones tempranas.
\end{itemize}

\end{frame}

\begin{frame}{Prueba con Usuarios: Correlaci\'on}

\begin{itemize}
    \item Errar es humano: Los usuarios prefieren probar (y fallar) a leer.
    \item Considerar la memoria del usuario al evaluar los resultados.
    \item Validar hip\'otesis triviales.
\end{itemize}

\end{frame}

\begin{frame}{Prueba con Usuarios: An\'alisis del Error Humano}

\begin{itemize}
    \item Evitar comandos de m\'as de 4 palabras.
    \item El contexto de los comandos no es lo (\'unico) que importa.
    \item Dar la debida importancia al entrenamiento del usuario.
    \item Considerar factores humanos como la fatiga.
    \item Reconsiderar los comandos con tasa de error elevada.
\end{itemize}

\end{frame}

\begin{frame}{Prueba con Usuarios: Encuesta}

\begin{itemize}
    \item Opini\'on positiva de los usuarios hacia \emph{TamTam Listens} y el reconocimiento del habla.
    \item Los puntos a mejorar est\'an en las respuestas de los usuarios.
    \item Preferir aplicaciones poco interactivas para una interfaz por voz.
\end{itemize}

\end{frame}

\section*{Trabajos Futuros}

\begin{frame}{Trabajos Futuros}

\begin{itemize}
    \uncover<+->{\item Utilizar TamTam Listens con personas con discapacidad.}
    \uncover<+->{\item Modelo Ac\'ustico Localizado.}
    \uncover<+->{\item Reducci\'on de ruido.}
    \uncover<+->{\item Optimizar generaci\'on de Modelo Ac\'ustico.}
    \uncover<+->{\item Integraci\'on con otros proyectos.}
\end{itemize}

\end{frame}

\section*{Publicaciones}

\begin{frame}{Publicaciones}

\begin{itemize}
    \item Art\'iculo aceptado en el simposio Inform\'atica y Sociedad de la 
        Conferencia Latinoamericana en Inform\'atica (CLEI) 2014.
        \newline \newline T\'itulo: \emph{Diseño de una Aplicación para Composición Musical utilizando Reconocimiento del Habla}.
\end{itemize}

\end{frame}


%%% BIBLIOGRAFIA %%%
\begin{frame}[allowframebreaks]
\frametitle{Referencias}
\printbibliography
\end{frame}

\section*{Anexos}

\section*{Aportes}

\begin{frame}{Aportes}

\begin{itemize}
    \uncover<+->{\item Fuente bibliogr\'afica importante como material para la introducci\'on al \'area.}
    \uncover<+->{\item Clasificaci\'on y criterios de selecci\'on de herramientas.}
    \uncover<+->{\item Dise\~no e implementaci\'on de una interfaz basada en voz para componer m\'usica.
    \begin{itemize}
            \item<+->{Gram\'atica para composici\'on musical.}
        \end{itemize}}
    \uncover<+->{\item Arquitectura de implementaci\'on: 
        \begin{itemize}    
            \item<+->{Basada en servicios.}
            \item<+->{Orientada a dispositivos de recursos limitados. Por ejemplo: XO del Proyecto Una Computadora Por Ni\~no.}
        \end{itemize}}
    \uncover<+->{\item Proceso de evaluaci\'on de la interfaz.}
    \uncover<+->{\item C\'odigo fuente.}
\end{itemize}
\end{frame}

% \begin{frame}[noframenumbering]{Anexos}
% \framesubtitle{Comparaci\'on de Herramientas}
% \begin{table}[H]
% \centering
% \footnotesize
% \begin{tabular}{|p{3.5cm}|>{\centering}p{3.5cm}|>{\centering}p{3.5cm}|>{\centering}p{3.5cm}|}
% \hline
%                                & Aplicaciones             &  APIs                            & Librer\'ias/\foreign{Frameworks} \tabularnewline
% \hline
% Conocimiento T\'ecnico         &     Bajo                    & Medio                            & Alto    \tabularnewline \hline
% Productividad                  &     Alto                    & Medio                            & Bajo    \tabularnewline \hline
% Flexibilidad                   &     Bajo                    & Medio                            & Alto    \tabularnewline \hline
% Alternativas  Propietarias      & \begin{itemize} \setlength{\itemsep}{1pt} \setlength{\parskip}{0pt} \setlength{\parsep}{0pt}\item Dragon Natural Speaking \end{itemize}  & \begin{itemize}  \setlength{\itemsep}{1pt} \setlength{\parskip}{0pt} \setlength{\parsep}{0pt} \item Web Speech API \item AT\&T Speech API \item Microsoft Speech API \item iSpeech API \item Dragon Mobile \end{itemize}  &  \tabularnewline \hline
% Alternativas de C\'odigo abierto & \begin{itemize}  \setlength{\itemsep}{1pt} \setlength{\parskip}{0pt} \setlength{\parsep}{0pt} \item Simon \item Palaver \end{itemize}          &           & \begin{itemize} \setlength{\itemsep}{1pt} \setlength{\parskip}{0pt} \setlength{\parsep}{0pt} \item Sphinx 4 \item PocketSphinx \item HTK \item Julius \item Kaldi \end{itemize} \tabularnewline
% \hline
% \end{tabular}
% \caption{Resumen general de las herramientas}
% \label{sec:resumen-herramientas}
% \end{table}
% \end{frame}

\begin{frame}[noframenumbering]{Anexos}
\framesubtitle{Aplicaciones}
\begin{table}[H]
\centering
\footnotesize
\begin{tabular}{|p{2.5cm}|p{2.5cm}|p{2.5cm}|p{2.5cm}|}
\hline
                                      &  Simon                                                       &  Palaver                                       & Dragon Naturally Speaking \\
\hline
Precio                                & Gratuito                                                     & Gratuito                                       & Desde 99\$  \\ \hline
Soporte para m\'ultiples idiomas      & Si                                                           & Si                                             & Si \\ \hline
Facilidad de configuraci\'on          & F\'acil                                    & Reducida                                       & F\'acil \\ \hline
Soporte para dispositivos m\'oviles   & Si                                                           & No                                             & Si, mediante otros productos \\
\hline
\end{tabular}
% \caption{Resumen de los criterios espec\'ificos de las aplicaciones}
\label{sec:resumen-aplicaciones}
\end{table}

\end{frame}

\begin{frame}[noframenumbering]{Anexos}
\framesubtitle{Interfaces de Programaci\'on de Aplicaciones (1/2)}
\begin{table}[H]
\centering
\footnotesize
\begin{tabular}{|p{2.5cm}|p{2.5cm}|p{2.5cm}|p{2.5cm}|}
\hline
                                      &  Web Speech API & AT\&T Speech api & Microsoft Speech API \\
\hline
Empresa o Instituci\'on responsable & \foreign{Speech API Community Group}. Implementado actualmente por \foreign{Google}  &  AT\&T Inc.  & Microsoft\\ \hline
Precio                              & Gratuito, a trav\'es de Google Chrome  & Un mill\'on de llamadas a la API por 99\$ anuales. 0.01\$ por llamada extra  & Gratuito\\ \hline
Soporte para m\'ultiples idiomas    & Si  & 2 idiomas & Si\\ \hline
Soporte Offline                     & No  & No  & Si \\ \hline
Dependencia de Plataforma           & No  & No & Si, solo para \emph{Microsoft Windows} \\
\hline
\end{tabular}
% \caption{Resumen de los criterios espec\'ificos de las APIs. Primera parte.}
\label{sec:resumen-apis}
\end{table}
\end{frame}

\begin{frame}[noframenumbering]{Anexos}
\framesubtitle{Interfaces de Programaci\'on de Aplicaciones (2/2)}
\begin{table}[H]
\centering
\footnotesize
\begin{tabular}{|p{2.5cm}|p{2.5cm}|p{2.5cm}|p{2.5cm}|}
\hline
                                      &  WAMI & iSpeech API & Dragon Mobile \\
\hline
Empresa o Instituci\'on responsable &  \pbox{2.5cm}{Instituto \\ Tecnol\'ogico de Massachusetts} & \foreign{iSpeech}  & \foreign{\pbox{2.5cm}{Nuance \\ Communications}} \\ \hline
Precio &  Gratuito  & Gratuito para aplicaciones no comerciales. De 0.02\$ a 0.0001\$ para otros usos & Gratuito para aplicaciones no comerciales. De 0.08\$ a 0.006\$ para otros usos. \\ \hline
Soporte para m\'ultiples idiomas  & No, aunque depende del reconocedor utilizado & Si & Si \\ \hline
Soporte Offline & No & No & No\\ \hline
Dependencia de Plataforma & No & No & No\\
\hline
\end{tabular}
% \caption{Resumen de los criterios espec\'ificos de las APIs. Segunda parte.}
\label{sec:resumen-apis-2}
\end{table}
\end{frame}

\begin{frame}[noframenumbering]{Anexos}
\framesubtitle{Librer{\'\i}as (1/2)}
\begin{table}[H]
\centering
\footnotesize
\begin{tabular}{|p{2.5cm}|p{2.5cm}|p{2.5cm}|p{2.5cm}|}
\hline
                                  &  Sphinx 4 & PocketSphinx & HTK \\
\hline
Empresa o Instituci\'on Responsable & \pbox{2.5cm}{Universidad \\ \foreign{Carnegie Mellon}} & \pbox{2.5cm}{Universidad \\ \foreign{Carnegie Mellon}} & Universidad de Cambridge \\ \hline
Precio & Gratuito & Gratuito & Gratuito \\ \hline
Licencia & BSD, con un componente privativo & BSD simplificada & C\'odigo fuente modificable, se prohibe redistribuci\'on.\\ \hline
Soporte para m\'ultiples idiomas & Si & Si & No\\ \hline
Dependencia de Plataforma & No & No & No \\ \hline
Modelos de Lenguaje Aceptados & JGSF, bigramas y trigamas &  JGSF, bigramas y trigamas &  JGSF, bigramas y trigamas \\ \hline
Modelos Ac\'usticos Aceptados & Modelo propio & Modelo propio &  Modelo propio \\ \hline
Uso de Memoria & No se recomienda para sistemas de poca memoria & Reducido en comparaci\'on a Sphinx 4 & Dependiente de la aplicaci\'on \\
\hline
\end{tabular}
% \caption[Resumen de los criterios espec\'ificos de las Librer\'ias/\foreign{Frameworks}.\protect\newline Primera parte.]
% {Resumen de los criterios espec\'ificos de las Librer\'ias/\foreign{Frameworks}. Primera parte.}
\label{sec:resumen-libs}
\end{table}


\end{frame}

\begin{frame}[noframenumbering]{Anexos}
\framesubtitle{Librer{\'\i}as (2/2)}
\begin{table}[H] 
\centering
\footnotesize
\begin{tabular}{|p{2.5cm}|p{2.5cm}|p{2.5cm}|}
\hline
                                  &  Julius & Kaldi \\
\hline
Empresa o Instituci\'on Responsable &  \foreign{Interactive Speech Technology Consortium} & Agencia Tecnol\'ogica de Rep\'ublica Checa \\ \hline
Precio & Gratuito & Gratuito \\ \hline
Licencia & BSD & Apache 2.0 \\ \hline
Soporte para m\'ultiples idiomas & No &  No \\ \hline
Dependencia de Plataforma & No & No \\ \hline
Modelos de Lenguaje Aceptados & Basados en gram\'aticas y modelos en formato ARPA & Modelos en formato FST \\ \hline
Modelos Ac\'usticos Aceptados & Modelo propio & Modelo propio \\ \hline
Uso de Memoria & Bajo & \\
\hline
\end{tabular}
% \caption[Resumen de los criterios espec\'ificos de las librer\'ias/\foreign{framework}s.\protect\newline Segunda parte.]{Resumen de los criterios espec\'ificos de las librer\'ias/\foreign{framework}s. Segunda parte.}
\label{sec:resumen-libs-2}
\end{table}
\end{frame}

\begin{frame}[noframenumbering]{Anexos}
\framesubtitle{Ecuaci\'on Fundamental del Reconocimiento del Habla}
\begin{align*}
\hat{W} = \argmax_{W \in L} P(W|O)
\end{align*}

Usando la Regla de Bayes puede reescribirse como:

\begin{align*}
\hat{W} = \argmax_{W \in L} \frac{P(O|W)P(W)}{P(O)}
\end{align*}

Se busca la oraci\'on con mayor probabilidad dada una entrada ac\'ustica,
la misma para todas las oraciones evaluadas.
En otras palabras, el t\'ermino $P(O)$ es independiente de $W$. Por tanto:

\begin{align*}
\hat{W} = \argmax_{W \in L} P(O|W)P(W)
\end{align*}

\end{frame}

\begin{frame}[noframenumbering]{Anexos}
\framesubtitle{Escala de Mel}
\begin{figure}[H]
\centering
\includegraphics[width=0.7\linewidth]{./graphics/mel_hz.png}
\end{figure} 
\end{frame}

\begin{frame}[noframenumbering]{Anexos}
\framesubtitle{Algoritmo de Viterbi (1/4)}
El algoritmo utiliza una matriz de probabilidades $viterbi$, donde cada celda $viterbi[i,t]$ 
contiene la probabilidad del mejor camino teniendo en cuenta las $t$ primeras observaciones y 
terminando en el estado $i$ del modelo.

\begin{align*}
    viterbi[i,t] = \displaystyle \underset{q_1,q_2,\ldots,q_{t - 1}}{max} P(q1,q2,\ldots,q_{t - 1},
        q_t = i,o_1,o_2,\ldots,o_t \mid \lambda)    
\end{align*} 

Para calcular los valores de $viterbi[i,t]$, el algoritmo de Viterbi asume la invariante de la 
programaci\'on din\'amica. Esto es, se asume que si el mejor camino para una secuencia de observaciones 
pasa por un estado $q_i$, entonces este camino incluye el mejor camino hasta $q_i$ inclusive. 


\begin{align*}
    viterbi[i,t] = \displaystyle \underset{i}{max} (viterbi[i,t-1]a_{i,j})b_j(o_t)
\end{align*}

\end{frame}

\begin{frame}[noframenumbering]{Anexos}
\framesubtitle{Algoritmo de Viterbi (2/4)}

\begin{algorithm}[H]
\footnotesize
\caption{Algoritmo de Viterbi} \label{viterbi}
\begin{algorithmic}[1]
\REQUIRE $observaciones$ de longitud $T$, $grafo\mbox{-}estados$.
\ENSURE $estados$, el mejor camino.
\STATE $num\mbox{-}estados \leftarrow$ CANTIDAD-DE-ESTADOS($grafo\mbox{-}estados$) 
\STATE Crear una matriz de probabilidades $viterbi[num\mbox{-}estados, T]$
\FOR{cada estado $s$ desde $0$ hasta $num\mbox{-}estados$}
    \STATE $viterbi[s,0] = \pi_s$
\ENDFOR
\FOR{cada paso $t$ desde $0$ hasta $T - 1$}
    \FOR{cada estado $s$ desde $0$ hasta $num\mbox{-}estados$}
        \FOR{cada transici\'on $s'$ desde s especificada por el $grafo\mbox{-}estados$}
        \STATE $nuevo\mbox{-}puntaje \leftarrow viterbi[s,t] * a[s,s'] * b_{s'}[o_t]$
        \IF{$viterbi[s',t+1] = 0 \parallel nuevo\mbox{-}puntaje > viterbi[s',t+1]$}
            \STATE $viterbi[s',t+1] \leftarrow nuevo\mbox{-}puntaje$
            \STATE $puntero\mbox{-}retroceso[s',t+1] \leftarrow s$
        \ENDIF  
        \ENDFOR
    \ENDFOR
\ENDFOR
\STATE $estados \leftarrow$ retroceso desde la celda con mayor valor en la \'ultima columna de $viterbi[]$
\COMMENT{Usando $puntero-retroceso$}.
\RETURN $estados$
\end{algorithmic}
\end{algorithm}
\end{frame}

\begin{frame}[noframenumbering]{Anexos}
\framesubtitle{Algoritmo de Viterbi (3/4)}
\begin{figure}[H]
\centering
\includegraphics[width=0.6\linewidth]{./graphics/hmm-viterbi2.png}
\end{figure}
\end{frame}

\begin{frame}[noframenumbering]{Anexos}
\framesubtitle{Algoritmo de Viterbi (4/4)}
\begin{figure}[H]
\centering
\includegraphics[width=1\linewidth]{./graphics/viterbi3.png}
\end{figure}
\end{frame}

\begin{frame}[noframenumbering]{Anexos}
\framesubtitle{Entrenamiento (1/4)}
%\subsection*{Modelo de Lenguaje}

El modelo se entrena contando las ocurrencias de cada n-grama en el corpus de texto, para luego
normalizar el conteo dividiendo sobre la cantidad total de n-gramas en el corpus.
A continuaci\'on se utilizan normalmente m\'etodos de reestimaci\'on, de modo a mejorar las estimaciones 
de n-gramas con un conteo muy bajo, o incluso igual a cero. Este proceso se conoce como suavizamiento
del modelo.

Finalmente, el conteo normalizado y suavizado de cada n-grama en el corpus del texto constituye su
probabilidad \cite{CollinsLanguage}.

\end{frame}

\begin{frame}[noframenumbering]{Anexos}
\framesubtitle{Entrenamiento (2/4)}

%\subsection*{HMM: Estructura del grafo de estados}
El conjunto de estados de cada HMM, $S$, y las transiciones entre estos estados se definen en base
al diccionario fon\'etico. A modo de ejemplo, \mbox{CMUdict \cite{CMUdict}} es un diccionario fon\'etico
que contiene correspondencias entre palabras y las secuencias de fonemas que las componen para m\'as de
125.000 palabras del idioma ingl\'es.

\begin{figure}[H]
\centering
\includegraphics[width=0.8\linewidth]{./graphics/diccionario.png}
\end{figure}
\end{frame}

\begin{frame}[noframenumbering]{Anexos}
\framesubtitle{Entrenamiento (3/4)}
Para cada HMM es necesario estimar los siguientes par\'ametros:
    \begin{itemize}
        \item Las probabilidades de estado inicial: $\pi_i$
        \item Las probabilidades de transici\'on: $a_{ij}$
        \item Las probabilidades de observaci\'on: $b_j(o_t)$ 
    \end{itemize}

Para esto se cuenta con \cite{Jurafsky}:
    \begin{itemize}
        \item  Un corpus de voz, compuesto por una colecci\'on de grabaciones de voz junto
        con sus transcripciones de texto.
        \item Un corpus de voz de menor tama\~no etiquetado fon\'eticamente. 
        Esto es, donde fragmentos de la se\~nal est\'an asociados a su fonema correspondiente.
    \end{itemize}

\end{frame}

\begin{frame}[noframenumbering]{Anexos}
\framesubtitle{Entrenamiento (4/4)}

Para las probabilidades de estado inicial, se asume que cualquier estado es igualmente probable 
como estado inicial. De manera similar, para las probabilidades de transici\'on se asume que, para cada estado, cualquier transici\'on a otro estado es igualmente probable.

Las probabilidades de observaci\'on se estiman inicialmente mediante el peque\~no corpus 
de voz etiquetado fon\'eticamente.

La siguiente etapa var{\'\i}a de acuerdo al m\'etodo escogido para la definici\'on de las probabilidades
de observaci\'on, $b_j$. Si se utilizan funciones Gaussianas, el cual constituye el caso m\'as
frecuente, las estimaciones iniciales se mejoran mediante el algoritmo de Baum--Welch.

El algoritmo de Baum--Welch calcula dos probabilidades adicionales en base a las estimaciones
iniciales de $a_{ij}$ y $b_j(o_t)$, denominadas probabilidad de avance y probabilidad de retroceso. 
Utilizando estas probabilidades, se mejoran las estimaciones $a_{ij}$ y $b_j(o_t)$ mediante
un proceso iterativo que se repite hasta que los valores converjan \cite{Rabiner89atutorial}.
\end{frame}

\begin{frame}[noframenumbering]{Anexos}
\framesubtitle{Algoritmo de Baum--Welch}
\begin{figure}[H]
\centering
\includegraphics[width=1\linewidth]{./graphics/forward-backward.png}
\end{figure}
\end{frame}

\begin{frame}[noframenumbering]{Anexos}
\framesubtitle{Dynamic Time Warping}
\begin{figure}[H]
\centering
\includegraphics[width=1\linewidth]{./graphics/dtw-anexos.png}
\end{figure}
\end{frame}

\begin{frame}[noframenumbering]{Anexos}
\framesubtitle{Tamaño de Muestra (1/2)}
We do have some useful advice from the research community that a study utilizing 8-25 participants per group is a sensible range to consider and that 10-12 participants is probably a good baseline range.
(\textit{How To Specify the Participant Group Size for Usability Studies: A Practitioner’s Guide}, Macefield, 2009)
\end{frame}


\begin{frame}[noframenumbering]{Anexos}
\framesubtitle{Tamaño de Muestra (2/2)}
\begin{figure}[H]
\centering
\includegraphics[width=1\linewidth]{./graphics/groupsize.png}
\caption{\textit{Beyond the five-user assumption: Benefits of
increased sample sizes in usability testing}, Faulkner, 2003}
\end{figure}
\end{frame}

\begin{frame}[noframenumbering]{Anexos}
\framesubtitle{Coeficientes de Correlaci\'on de Pearson}
El coeficiente de Pearson es una medida del grado de correlaci\'on lineal o dependencia entre dos 
variables $X$ e $Y$. El valor del coeficiente se encuentra entre -1 y 1 inclusive. 
El valor -1 indica que las variables est\'an correlacionadas negativamente 
(cuando $X$ crece, $Y$ decrece y viceversa), 0 indica que no existe correlaci\'on y 1 que existe una 
correlaci\'on positiva (cuando $X$ crece, $Y$ crece).
\end{frame}

\begin{frame}[noframenumbering]{Anexos}
\framesubtitle{Satisfacci\'on del Usuario (1/2)}
La opini\'on del usuario se mide a trav\'es de los resultados de la encuesta que se realiza como parte de
la prueba de usabilidad, posterior a la finalizaci\'on de las tareas. Las respuestas se
registran utilizando una escala de Likert con valores del 1 al 7.
\end{frame}

\begin{frame}[noframenumbering]{Anexos}
\framesubtitle{Satisfacci\'on del Usuario (2/2)}
Posteriormente, de modo a eliminar el potencial efecto de los estilos de respuesta
de los usuarios, se utiliza el m\'etodo de estandarizaci\'on 
por rango para reescalar los resultados de la encuesta.

Siendo:
\begin{itemize}
	\item $min_i$ la respuesta de menor valor del usuario $i$.
	\item $max_i$ la respuesta de mayor valor del usuario $i$.
\end{itemize}

Para cada respuesta $s$ del usuario $i$, el valor ajustado $s_a$ se define como:

\begin{equation*}
s_a=\frac{s-min_i}{max_i-min_i}
\end{equation*}


De esta manera, todas las respuestas pasan a estar en el rango entre 0 y 1.  
\end{frame}


\end{document}
