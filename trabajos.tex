\section{Trabajos Futuros}

\begin{frame}{Trabajos Futuros}
Entre los trabajos futuros que podr{\'\i}an realizarse de modo a profundizar en el \'area de reconocimiento del habla
pueden mencionarse:

\begin{itemize}
\item{\textbf{Utilizar TamTam Listens con personas con \\discapacidad\\}}
Habiendo obtenido resultados satisfactorios en la prueba con usuarios realizada, el siguiente paso podr{\'\i}a ser
una prueba con personas con discapacidades, para quienes la interfaz mediante voz podr{\'\i}a resultar
muy beneficiosa. 

\item{\textbf{Modelo Ac\'ustico Localizado\\}} 
La interfaz implementada utiliza el prove\'ido por el proyecto \emph{VoxForge}. 
Se podr\'ia buscar utilizar un modelo ac\'ustico propio con el acento caracter\'istico de nuestro pa{\'\i}s
en busca de una mejora en la precisi\'on del sistema.
\end{itemize}
\end{frame}

\begin{frame}{Trabajos Futuros (2)}
\begin{itemize}
\item{\textbf{Reducci\'on de ruido\\}}
El ruido ambiente es un problema que afecta al reconocimiento del habla y la precisi\'on de la soluci\'on 
desarrollada en este trabajo es considerablemente sensible al mismo. Se podr\'ian investigar los algoritmos
de reducci\'on de ruido disponibles y ver su aplicabilidad al software desarrollado.

\item{\textbf{Optimizar generaci\'on de Modelo Ac\'ustico\\}}
El entrenamiento de un modelo ac\'ustico es un proceso pesado por el potencial volumen de datos que debe
manejarse de modo a lograr resultados satisfactorios en cuanto a precisi\'on.
Es posible plantear la utilizaci\'on de de \emph{Grid Computing} de modo a distribuir y optimizar esta tarea.

\end{itemize}
\end{frame}

\begin{frame}{Trabajos Futuros (3)}
\begin{itemize}

\item{\textbf{Integraci\'on con otros proyectos\\}}
La integraci\'on del reconocimiento del habla a proyectos existentes puede dar lugar a interesentes trabajos.
A modo de ejemplo, podr{\'\i}an implementarse interfaces mediante voz para el robot educativo uruguayo 
Buti\'a \cite{Butia} o el autom\'ovil el\'ectrico paraguayo Aguar\'a.
\end{itemize}
\end{frame}
