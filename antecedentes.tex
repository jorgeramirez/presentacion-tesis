\section{Antecedentes del Reconocimiento del Habla}

\begin{frame}{Antecedentes del Reconocimiento del Habla. 50s}
La investigaci\'on en el \'area se inicia en los a\~nos 50.
Los primeros reconocedores se basaron en la correspondencia de patrones \cite{DavisAutomatic1952},
en especial tras el desarrollo del \emph{Dynamic Time Warping}, a finales de los
60 \cite{VintsyukSpeech1968}.
\end{frame}

\begin{frame}{Antecedentes del Reconocimiento del Habla. 70s}
Los a\~nos 70 fueron de gran importancia para el \'area,
debido a la publicaci\'on de la teor{\'\i}a b\'asica de los modelos ocultos de Markov (HMM por sus siglas en ingl\'es) 
en una serie de art{\'\i}culos de Baum y sus colegas \cite{Rabiner89atutorial}.
\end{frame}

\begin{frame}{Antecedentes del Reconocimiento del Habla. 80s}
La d\'{e}cada de los 80 fue caracterizada por la adopci\'on del modelo estad{\'\i}stico en el proceso
de reconocimiento del habla. Cabe destacar tambi\'en el
desarrollo del modelo de lenguaje basado en N-gramas en IBM \cite{JelinekTheDevelopment1986}.
\end{frame}

\begin{frame}{Antecedentes del Reconocimiento del Habla. 90s y actualidad.}
A partir de la d\'ecada de los 90, la combinaci\'on de los avances relacionados al 
reconocimiento del habla con el incremento de poder computacional y de almacenamiento result\'o 
en la llegada del reconocimiento del habla al usuario com\'un. Como ejemplos pueden citarse
las aplicaciones de \emph{Dragon Systems} \cite{BarnettMultilingual1996} y, m\'as tarde,
\emph{Google Now} \cite{GoogleNow} y el asistente personal de \emph{Apple}, 
\emph{Siri} \cite{AppleSiri}.

Seg\'un los reportes de la consultora tecnológica Gartner correspondientes al año 2013,
el reconocimiento del habla se convertir\'a en una tecnolog{\'\i}a estable y con beneficios
ampliamente demostrados en los pr\'oximos 2 a 5 a\~nos \cite{Gartner2013}.
\end{frame}
